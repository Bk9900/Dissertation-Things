\documentclass[12pt,a4paper]{article}
\usepackage[utf8]{inputenc}
\usepackage[T1]{fontenc}
\usepackage{amsmath}
\usepackage{amsfonts}
\usepackage{amssymb}
\usepackage{graphicx}
\usepackage{geometry}
\usepackage{mathrsfs}
\usepackage{braket}
\usepackage{simpler-wick}
\usepackage{simplewick}
\usepackage{tikz}
\usepackage{tikz-feynman}
\usepackage{subcaption}
\usepackage{bm}
\usepackage{slashed}
\usepackage{tensor}
\usepackage{hyperref}
\usepackage{mathtools}
%\usepackage{pst-node}
%\usepackage{auto-pst-pdf}
\title{Quantum Error Correction - Notes}
\author{Ben Karsberg}
\date{2021-22}
\newgeometry{vmargin={15mm}, hmargin={20mm,20mm}}
\numberwithin{equation}{section}
\newcommand{\ketbra}[2]{\ket{#1}\bra{#2}}
\newcommand{\ketbras}[1]{\ketbra{#1}{#1}}
\newcommand{\Pc}{P_{\text{code}}}
\begin{document}
	\maketitle
	\section{Week 2: Harlow Toy Model}
	\begin{itemize}
		\item References:
		\begin{itemize}
			\item https://arxiv.org/pdf/1607.03901.pdf (Harlow paper)
			\item https://pos.sissa.it/305/002/pdf (Harlow presentation)
			\item https://arxiv.org/pdf/quant-ph/9901025.pdf\#page=5\&zoom=100,0,0 (Quantum secret sharing)
		\end{itemize}
		\item Harlow presents a toy model of error correction, which we go through
		\item Suppose we wish to protect the qutrit state
		\begin{equation}
			\ket{\psi}=\sum_{i=0}^{2}a_{i}\ket{i}
		\end{equation}
		against erasure
		\item We project to the code subspace spanned by logical qutrits
		\begin{equation}
			\begin{aligned}
				\ket{0_{L}}&=\frac{1}{\sqrt{3}}(\ket{000}+\ket{111}+\ket{222})\\
				\ket{1_{L}}&=\frac{1}{\sqrt{3}}(\ket{012}+\ket{120}+\ket{201})\\
				\ket{2_{L}}&=\frac{1}{\sqrt{3}}(\ket{021}+\ket{102}+\ket{210})
			\end{aligned}
		\end{equation}
		\item There's two points of note here:
		\begin{enumerate}
			\item This code subspace is symmetric under cyclic permutations of the physical qubits
			\item Each individual qutrit is maximally mixed - it is equal parts $\ket{0}$, $\ket{1}$, $\ket{2}$ (this means it functions as a \textit{quantum secret-sharing code} too)
		\end{enumerate}
		\item To actually prepare these logical qutrits, we adjoin two ancillary qutrits in the state
		\begin{equation}
			\ket{\chi}_{23}=\frac{1}{\sqrt{3}}(\ket{00}+\ket{11}+\ket{22})
		\end{equation}
		\item We define a unitary operation $U_{12}$ on the first two qutrits via the permutation
		\begin{equation}
			\begin{aligned}
				\ket{00}&\to\ket{00} & \ket{11}&\to\ket{01} & \ket{22}&\to\ket{02}\\
				\ket{01}&\to\ket{12} & \ket{12}&\to\ket{10} & \ket{20}&\to\ket{11}\\
				\ket{02}&\to\ket{21} & \ket{10}&\to\ket{22} & \ket{21}&\to\ket{20}
			\end{aligned}
		\end{equation}
		\item Then, note that this unitary implements
		\begin{equation}
			\ket{i_{L}}=U_{12}(\ket{i}_{1}\otimes\ket{\chi}_{23})
		\end{equation}
		for all physical basis elements $\ket{i}$
		\item So, the state can be encoded just by doing
		\begin{equation}
			\ket{\psi_{L}}=U_{12}(\ket{\psi}_{1}\otimes\ket{\chi}_{23})
		\end{equation}
		\item We can similarly encode with a unitary on the (1,3) or (2,3) subsystems by symmetry
		\item This explicitly provides the erasure protection procedure: given only two subsystems of $\ket{\psi_{L}}$, we can just apply the corresponding $U_{ij}^{\dagger}$ and we get back the original $\ket{\psi}$ in one of the subsystems
		\item Sticking with Harlow's numbering, suppose the third qutrit is erased, so we recover $\ket{\psi}$ perfectly in the first qutrit
		\item The no-cloning theorem then implies that we should not be able to work out \textbf{any} information about $\ket{\psi}$ given knowledge of just the third qutrit alone
		\item This is quite obvious from (1.6) and from (1.2): the third qutrit in the encoded state is always maximally mixed/proportional to the identity, so contains zero information about the unencoded state on its own as expected
		\item What can we say about the state $\ket{\chi}$?
		\item All we can really say is that $\ket{\chi}$ has to be entangled
		\item Suppose $\ket{\chi}=\ket{\alpha}\otimes\ket{\beta}$, so is in a product state; then, (1.6) becomes
		\begin{equation}
			\ket{\psi_{L}}=U_{12}(\ket{\psi}_{1}\otimes\ket{\alpha}_{2}\otimes\ket{\beta}_{3})
		\end{equation}
		so the third qutrit is sitting in the state $\ket{\beta}$ \textbf{for all} states in the code subspace, so it provides no info about $\ket{\psi}$ which the second qubit didn't already know 
		\item This can also be viewed under the lens of \textit{quantum secret sharing} - Harlow's toy model is a $((2,3))$ threshold scheme, so we should be able to recover full state information from any two qubits, but \textbf{no} information from any one share
		\item $\ket{\chi}$ clearly needs entanglement for the former to hold
		\item This can equivalently be phrased by saying that a $U_{23}$ and $U_{13}$ can't both simultaneously exist if $\ket{\chi}$ is a product state 
		\item This correctability can also be framed in terms of \textit{logical operators}
		\item Consider the operator on a physical qutrit as
		\begin{equation}
			O\ket{i}=\sum_{j}(O)_{ji}\ket{j}
		\end{equation}
		\item We can (in some sense) project this operator up into the code subspace by just defining an operator with the same matrix elements $(O)_{ij}$ acting on the basis for the code subspace:
		\begin{equation}
			O_{L}\ket{i_{L}}=\sum_{j}(O)_{ji}\ket{j_{L}}
		\end{equation}
		\item We can then just arbitrarily define the action of this logical operator on the full 3-qutrit space which the code subspace is a subspace of
		\item Now, consider the case where $O=O_{1}$ acts only on the first qutrit in the toy model
		\item We can define a logical operator on the code subspace which behaves equivalently to $O_{1}$ but which has non-trivial support only on the first and second qutrits:
		\begin{equation}
			O_{12}\coloneqq U_{12}O_{1} U_{12}^{\dagger}
		\end{equation}
		\item Thus any operator acting on a single qutrit can be represented as a logical operator with support on any two of the qutrits
		\item Harlow says that this is an analogue of \textit{subregion duality} in AdS/CFT
		\item He also says it is an analogue of \textit{radial commutativity}, meaning that any logical operator $O_{L}$ on $\mathcal{H}_{\text{code}}$ commutes with any operator acting on a single qutrit
		\item But $O_{12}$ clearly commutes with any operator $X_{3}$ acting on qutrit 3, and similar for $O_{13}$ and $O_{23}$; since all of these act identically to $O_{L}$ on $\mathcal{H}_{\text{code}}$, it must be that $O_{L}$ commutes with all single qutrit operators
		\item More concretely, for any two states $\ket{\psi_{L}}$ and $\ket{\phi_{L}}$ in $\mathcal{H}_{\text{code}}$, we have
		\begin{equation}
			\braket{\psi_{L}|[O_{L},X]|\phi_{L}}=0
		\end{equation}
		where $X$ is any single qutrit operator
		\item A version of the \textit{Ryu-Takayanagi formula} also holds in this code
		\item This formula is
		\begin{equation}
			S(\rho_{A})=\text{Tr}(\rho\mathcal{L}_{A})+S_{bulk}(\rho_{\mathcal{E}_{A}})
		\end{equation}
		\item To see the link, suppose we have an arbitrary mixed state $\rho_{L}$ on $\mathcal{H}_{\text{code}}$, which is the encoding of a physical mixed state $\rho$
		\item If we define 
		\begin{equation}
			\rho_{1}=\sum_{i,j}a_{i,j}\ketbra{i}{j}
		\end{equation}
		on the first subsystem, then by (1.5) we have that
		\begin{equation}
			\rho_{L}=U_{12}(\rho_{1}\otimes\ketbras{\chi}_{23})
		\end{equation}
		\item Define $\rho_{L}^{3}\coloneqq\text{Tr}_{12}(\rho_{L})$ and $\rho_{L}^{12}\coloneqq\text{Tr}_{3}(\rho_{L})$; we then explicitly compute
		\begin{equation}
			\begin{aligned}
				\rho_{L}^{3}&=\frac{1}{3}(\ketbras{0}+\ketbras{1}+\ketbras{2})=\frac{1}{3}I_{3}\\
				\rho_{L}^{12}&=U_{12}\left(\rho_{1}\otimes\frac{1}{3}I_{2}\right)U^{\dagger}
			\end{aligned}
		\end{equation}
		\item This gives von Neumann entropies
		\begin{equation}
			\begin{aligned}
				S(\rho_{L}^{3})&=-3\times\frac{1}{3}\log\frac{1}{3}=\log{3}\\
				S(\rho_{L}^{12})&=S(\rho_{1})+S\left(\frac{1}{3}I_{2}\right)=S(\rho_{L})+\log{3}
			\end{aligned}
		\end{equation}
		\item In this case, (1.12) holds: `the logical degree of freedom in the bulk is not in the entanglement wedge of the third physical qutrit, but is in the entanglement wedge of the union of the first and second'
		\item Thus we expect it contributes entropy only to $S(\rho_{L}^{12})$, which is what we observe
		\item WHAT DOES THIS MEAN AAAAA
	\end{itemize}
	\section{Stabiliser Codes}
	\subsection{The Stabiliser Formalism}
	\begin{itemize}
		\item Consider the EPR state
		\begin{equation}
			\ket{\psi}=\frac{1}{\sqrt{2}}(\ket{00}+\ket{11})
		\end{equation}
		\item This state satisfies $X_{1}X_{2}\ket{\psi}=\ket{\psi}$ and $Z_{1}Z_{2}\ket{\psi}=\ket{\psi}$; we say it is \textit{stabilised} by $X_{1}X_{2}$ and $Z_{1}Z_{2}$
		\item It's easy to check that in fact $\ket{\psi}$ is the unique state stabilised by these operators
		\item The idea of the formalism is that it's often easier to work with the operators which stabilise a state rather than the state itself
		\item The key to the stabiliser formalism is lots of group theory, with the group of principal interest being the \textit{Pauli group} $G_{n}$ on $n$ qubits
		\item This is defined on a single qubit as consisting of all the Pauli matrices together with multiplicative factors of $\pm 1$ and $\pm i$:
		\begin{equation}
			G_{1}\coloneqq\{\pm I,\pm iI, \pm X, \pm iX, \pm Y, \pm iY, \pm Z, \pm iZ\}
		\end{equation}
		\item The Pauli group on $n$ qubits is then just the $n$ fold tensor product of any elements of $G_{1}$
		\item Suppose $S\leq G_{n}$ is a subgroup, and define $V_{S}$ as the set of $n$ qubit states fixed by $S$: $\ket{\psi}\in V_{S} \implies S\ket{\psi}=\ket{\psi}$
		\item $V_{S}$ is called the \textit{vector space stabilised by} $S$, and $S$ is the \textit{stabiliser} of $V_{S}$
		\item As an example, consider $n=3$ and $S=\{I,Z_{1}Z_{2},Z_{2}Z_{3},Z_{1}Z_{3}\}$
		\item $Z_{1}Z_{2}$ fixes $\text{Span}(\ket{000},\ket{001},\ket{110},\ket{111})$, and $Z_{2}Z_{3}$ fixes $\text{Span}(\ket{000},\ket{100},\ket{011},\ket{111})$
		\item Note that $\ket{000}$ and $\ket{111}$ are in both these lists, so $V_{S}=\text{Span}(\ket{000},\ket{111})$
		\item We determined $V_{S}$ by looking at the subspaces which were stabilised by just two operators here, which is a more general example of using generators to describe a group
		\item We write $G=\braket{g_{1},\ldots,g_{l}}$ for the group generated by the $l$ elements above; in this example, $S=\braket{Z_{1}Z_{2},Z_{2}Z_{3}}$ since $Z_{1}Z_{3}=(Z_{1}Z_{2})(Z_{2}Z_{3})$ and $I=(Z_{1}Z_{2})^{2}$
		\item So to find $V_{S}$, we only need to find the states stabilised by the generators
		\item Not every subgroup of the Pauli group is the stabiliser for a non-trivial vector space
		\item For example, consider $\{\pm I, \pm X\}\leq G_{1}$; the only solution to $-I\ket{\psi}=\ket{\psi}$ is $\ket{\psi}=0$
		\item It turns out that necessary and sufficient conditions for a subgroup $S$ to stabilise a non-trivial space are
		\begin{enumerate}
			\item All elements of $S$ commute
			\item $-I\notin S$
		\end{enumerate}
		\item We can use the stabiliser formalism to describe error correcting codes
		\item For example, the 7 qubit Steane code has code space stabilised by the 7 generators
		\begin{equation}
			\begin{aligned}
				g_{1}&=X_{4}X_{5}X_{6}X_{7}\\
				g_{2}&=X_{2}X_{3}X_{6}X_{7}\\
				g_{3}&=X_{1}X_{3}X_{5}X_{7}\\
				g_{4}&=Z_{4}Z_{5}Z_{6}Z_{7}\\
				g_{5}&=Z_{2}Z_{3}Z_{6}Z_{7}\\
				g_{6}&=Z_{1}Z_{3}Z_{5}Z_{7}
			\end{aligned}
		\end{equation}
		\item A useful way of presenting generators $g_{1},\ldots,g_{l}$ is via the \textit{check matrix}
		\item This is an $l\times 2n$ matrix whose rows correspond to the generators, and the LHS contains 1s to indicate which generators contain $X$s and the RHS for $Z$s, so a 1 on both sides indicates a $Y$
		\item We usually use $r(g)$ to denote the $2n$-dimensional row vector rep of an element $g$ of the Pauli group
		\item Define a $2n\times 2n$ matrix by
		\begin{equation}
			\Lambda=\begin{pmatrix}0&I\\I&0\end{pmatrix}
		\end{equation}
		and then elements $g$ and $g'$ of $G_{n}$ commute iff $r(g)\Lambda r(g')^{T}=0$ modulo 2
		\item We have the following proposition: let $S=\braket{g_{1},\ldots,g_{l}}$ be such that $-I$ is not in $S$. Then, the generators are independent iff the rows of the corresponding check matrix are linearly independent
		\item An even more useful proposition is that if $S$ is as above where all generators are independent, and if we fix $i\in\{1,\ldots, l\}$, then $\exists g\in G_{n}$ such that $gg_{i}g^{\dagger}=-g_{i}$ and $gg_{j}g^{\dagger}=g_{j}$ for all $j\neq i$
		\item This proposition is leveraged to prove that if $S=\braket{g_{1},\ldots, g_{n-k}}$ is generated by $n-k$ independent and commuting elements from $G_{n}$, and $-I\notin S$, then $V_{S}$ is a $2^{k}$-dimensional vector space
	\end{itemize}
	\subsection{Unitary Gates}
	\begin{itemize}
		\item Suppose we apply a unitary $U$ to $V_{S}$ stabilised by $S$
		\item For any $\ket{\psi}\in V_{S}$ and $g\in S$, we have
		\begin{equation}
			U\ket{\psi}=Ug\ket{\psi}=UgU^{\dagger}U\ket{\psi}
		\end{equation}
		meaning $U\ket{\psi}$ is stabilised by $UgU^{\dagger}$
		\item This means that $UV_{S}$ is stabilised by $USU^{\dagger}$
		\item Moreover, if $S$ is generated by $g_{1},\ldots, g_{l}$, then $USU^{\dagger}$ is generated by $Ug_{1}U^{\dagger},\ldots$, so we only need to see how this unitary affects the generators
		\item For certain $U$s, the generators transform particularly nicely
		\item Suppose for example we apply a Hadamard to a single qubit, and note that
		\begin{equation}
			HXH^{\dagger}=Z,\quad HYH^{\dagger}=-Y,\quad HZH^{\dagger}=X
		\end{equation}
		\item So, imagine we have $n$ qubits in a state stabilised by $\braket{Z_{1},\ldots,Z_{n}}$ (this is clearly just $\ket{0}^{\otimes n}$)
		\item Applying the Hadamard to each qubit in turn, the resultant state is stabilised by $\braket{X_{1},\ldots,X_{n}}$, being $\ket{+}^{\otimes n}$
		\item The cool thing here is that we need $2^{n}$ amplitudes to specify the final state, but only $n$ generators
		\item This isn't that surprising: after applying the Hadamards, the resultant state is still not entangled so it's not too surprising that we can obtain a compact description
		\item However, let's consider the controlled not gate
		\item Denote $U$ the controlled not with qubit 1 as control; then, a bit of algebra shows
		\begin{equation}
			UX_{1}U^{\dagger}=X_{1}X_{2}
		\end{equation}
		\item Similar calculations show $UX_{2}U^{\dagger}=X_{2}$, $UZ_{1}U^{\dagger}=Z_{1}$, and $UZ_{2}U^{\dagger}=Z_{1}Z_{2}$
		\item So we can describe CNOT and Hadamards through the stabiliser formalism
		\item We can also describe the phase gate, which is the operation 
		\begin{equation}
			S=\begin{pmatrix}1&0\\0&i\end{pmatrix}
		\end{equation}
		\item We find
		\begin{equation}
			SXS^{\dagger}=Y,\quad SZS^{\dagger}=Z
		\end{equation}
		\item It turns out that \textbf{any} operation taking $G_{n}\to G_{n}$ under conjugation can be written as some composition of Hadamards, CNOTS, and phase gates
		\item We call the set of $U$ such that $UG_{n}U^{\dagger}=G_{n}$ the \textit{normaliser} of $G_{n}$, denoted $N(G_{n})$
		\item We claim therefore that $N(G_{n})$ is generated by Hadamards, CNOTs, and phase gates
		\item Unfortunately, most gates are not in $N(G_{n})$, in particular the $\pi/8$ gate and Toffoli gate
		\item This makes analysing quantum circuits involving these much harder in the stabiliser formalism
		\item It turns out though that error correction can be done without these, so analysing stabiliser codes can be made very nice in this formalism
	\end{itemize}
	\subsection{Measurement}
	\begin{itemize}
		\item Measurements can also be described within the stabiliser formalism
		\item Suppose we measure $g\in G_{n}$, and assume $g$ is a product of Pauli matrices with no factor of $-1$ or $\pm i $ out the front
		\item Assume further that the system is in state $\ket{\psi}$ with stabiliser $\braket{g_{1},\ldots,g_{n}}$; how does this stabiliser transform under the measurement?
		\item There's two cases:
		\begin{enumerate}
			\item $g$ commutes with all generators of the stabiliser
			\item $g$ anticommutes with one or more of the generators of the stabiliser
		\end{enumerate}
		\item In case 1, either $g$ or $-g$ is in the stabiliser: since $g_{j}g\ket{\psi}=gg_{j}\ket{\psi}=g\ket{\psi}$ for each stabiliser generator, $g\ket{\psi}\in V_{S}$ and so is proportional to $\ket{\psi}$
		\item Since $g^{2}=I$, it follows that $g\ket{\psi}=\pm\ket{\psi}$, so $g$ or $-g$ is in the stabiliser
		\item Assuming $g$ is in the stabiliser ($-g$ is analogous), $g\ket{\psi}=\ket{\psi}$, so a measurement of $g$ gives $+1$ with probability 1
		\item Therefore the measurement does not disturb the system, and leaves the stabiliser invariant
		\item For the second case where $g$ anticommutes with $g_{1}$ and commutes with all other generators, we note the following
		\item $g$ has eigenvalues $\pm1$, so the projectors onto the $\pm1$ eigenspaces are given by $(I\pm g)/2$ respectively, and so the measurement probabilities are
		\begin{equation}
			\mathbb{P}(+1)=\text{Tr}\left(\frac{I+g}{2}\ketbras{\psi}\right),\quad\mathbb{P}(-1)=\text{Tr}\left(\frac{I-g}{2}\ketbras{\psi}\right)
		\end{equation}
		\item But, since $g_{1}\ket{\psi}=\ket{\psi}$ and $gg_{1}=-g_{1}g$, we find that
		\begin{equation}
			\mathbb{P}(+1)=\text{Tr}\left(\frac{I+g}{2}g_{1}\ketbras{\psi}\right)=\text{Tr}\left(g_{1}\frac{I-g}{2}\ketbras{\psi}\right)=\mathbb{P}(-1)
		\end{equation}
		and so we deduce $\mathbb{P}(\pm1)=1/2$
		\item Suppose we measure +1, so the system collapses to $\ket{\psi^{+}}=(I+g)\ket{\psi}/\sqrt{2}$, which has stabiliser $\braket{g,g_{2},\ldots,g_{l}}$, and similarly for $-1$ with $g\mapsto -g$
	\end{itemize}
	\subsection{Gottesman-Knill Theorem}
	\begin{itemize}
		\item We have implicitly proved so far the following
		\item Suppose a quantum computation is performed only involving states prepared in the computational basis, Hadamard gates, phase gates, CNOT gates, Pauli gates, and measurements of Pauli observables, together with classical operations conditioned on the outcome of measurements. Such a computation can be efficiently simulated on a classical computer
		\item This shows that some quantum computations involving highly entangled states can be simulated efficiently classically
	\end{itemize}
	\subsection{Stabiliser Codes}
	\begin{itemize}
		\item An $[n,k]$ \textit{stabiliser code} is defined to be the vector space $V_{S}$ stabilised by $S\leq G_{n}$ such that $-I\notin S$ and $S$ has $n-k$ independent, commuting generators $S=\braket{g_{1},\ldots, g_{n-k}}$
		\item We denote this code $C(S)$ in what follows
		\item What are the logical basis states for $C(S)$?
		\item A systematic way of choosing them is as follows; we could however just arbitrarily choose any $2^{k}$ orthonormal vectors in $C(S)$, that said
		\item Choose operators $\bar{Z}_{1},\ldots, \bar{Z}_{k}\in G_{n}$ such that $g_{1},\ldots, g_{n-k},\bar{Z}_{1},\ldots, \bar{Z}_{k}$ forms an independent and commuting set
		\item The $\bar{Z}_{j}$ operator plays the role of a logical Pauli $\sigma_{z}$ operator on logical qubit $j$, so the logical computational basis state $\ket{x_{1},\ldots, x_{k}}_{L}$ is defined to be the state with stabiliser
		\begin{equation}
			\braket{g_{1},\ldots, g_{n-k},(-1)^{x_{1}}\bar{Z}_{1},\ldots, (-1)^{x_{k}}\bar{Z}_{k}}
		\end{equation}
		\item Similarly, we define $\bar{X}_{j}$ to be the product of Pauli gates which take $\bar{Z}_{j}\mapsto -\bar{Z}_{j}$ under conjugation, and leaves all other $\bar{Z}_{i}$ and $g_{i}$ alone when acting by conjugation
		\item $\bar{X}_{j}$ has the effect of a NOT gate acting on the $j$th encoded qubit
		\item Moreover, $\bar{X}_{j}$ satisfies $\bar{X}_{j}g_{k}\bar{X}_{j}^{\dagger}=g_{k}$, and so commutes with the stabiliser
		\item $\bar{X}_{j}$ also commutes with all $\bar{Z}_{i}$ for $i\neq j$, and anticommutes when $i=j$
		\item How are error correcting properties of $C(S)$ related to the generators of the stabiliser?
		\item Suppose we encode a state by a $[n,k]$ stabiliser code $C(S)$ with generators as above, and an error $E\in G_{n}$ occurs
		\item When $E$ anticommutes with an element of the stabiliser, $E$ takes $C(S)$ to an orthogonal subspace so the error can be detected by an appropriate projective measurement
		\item If $E\in S$, then we don't have to worry since $E$ doesn't corrupt the space at all
		\item If $E$ commutes with all elements of $S$ but $E\notin S$, then we have issues
		\item The set of $E\in G_{n}$ such that $Eg=gE$ for all $g\in S$ is called the \textit{centraliser} of $S$ in $G_{n}$, denoted $Z(S)$
		\item For stabiliser groups $S$ which we concern ourselves with, the centraliser coincides with the normaliser of $S$, $N(S)$, which is just all elements $E\in G_{n}$ such that $EgE^{\dagger}\in S$ for all $g\in S$
		\item These observations motivate the error correction conditions for stabiliser codes
		\item Let $S$ be the stabiliser for a stabiliser code $C(S)$. Suppose $\{E_{j}\}$ is a set of operators in $G_{n}$ such that $E_{j}^{\dagger}E_{k}\notin N(S)-S$ for all $j,k$. Then $\{E_{j}\}$ is a correctable set of errors for $C(S)$
		\item WLOG we can restrict to considering $E_{j}\in G_{n}$ such that $E_{j}^{\dagger}=E_{j}$
		\item This doesn't tell us explicitly how to perform the error correction operation
		\item Suppose $\{g_{1},\ldots, g_{n-k}\}$ is a set of generators for the stabiliser of a $[n,k]$ stabiliser code, and $\{E_{j}\}$ is a set of correctable errors
		\item Error detection is performed by measuring the generators of the stabiliser $g_{1}$ to $g_{n-k}$ in turn, to obtain the error syndrome, which consists of the results of the measurements $\beta_{1}$ to $\beta_{n-k}$
		\item If $E_{j}$ occurred, then the syndrome is given by $\beta_{l}$ such that $E_{j}g_{l}E_{j}^{\dagger}=\beta_{l}g_{l}$
		\item If $E_{j}$ is the unique error having this syndrome, recovery is just performed by applying $E_{j}^{\dagger}$
		\item If there are two errors $E_{j}$, $E_{j'}^{\dagger}$ given rise to the same syndrome, it follows that $E_{j}PE_{j}^{\dagger}=E_{j'}PE_{j'}^{\dagger}$ where $P$ is the projector onto the code space, so $E_{j}^{\dagger}E_{j'}PE_{j'}^{\dagger}E_{j}=P$, so $E_{j}^{\dagger}E_{j'}\in S$, so applying $E_{j}^{\dagger}$ after the error $E_{j'}$ occurs results in a successful recovery
		\item We define the \textit{weight} of an error $E\in G_{n}$ as the number of terms in the tensor product which aren't equal to the identity
		\item So e.g. the weight of $X_{1}Z_{4}Y_{8}$ is three
		\item The \textit{distance} of $C(S)$ is defined to be the minimum weight of an element of $N(S)-S$, and if $C(S)$ is an $[n,k]$ code with distance $d$, we say that $C(S)$ is an $[n,k,d]$ stabiliser code
		\item By the error correcting conditions, a code with distance at least $2t+1$ is able to correct arbitrary errors on any $t$ qubits
	\end{itemize}
	\subsection{Examples}
	\subsubsection{Bit-Flip Code}
	\begin{itemize}
		\item Consider the three qubit bit-flip code, with code subspace spanned by $\ket{000}$ and $\ket{111}$
		\item The stabiliser is therefore $S=\braket{Z_{1}Z_{2},Z_{2}Z_{3}}$
		\item By inspection, every possible product of two elements from the set of errors $\{I,X_{1},X_{2},X_{3}\}$ anticommutes with at least one of the generators of $S$, so by the error correcting conditions the set $\{I,X_{1},X_{2},X_{3}\}$ forms a correctable set of errors
		\item Error detection is effected by measuring the stabiliser generators
		\item If e.g. $X_{1}$ occurred, then the stabiliser becomes $\braket{-Z_{1}Z_{2},Z_{2}Z_{3}}$, so the syndrome measurement gives the results $\pm 1$
		\item Similarly, $X_{2}$ gives $-1$ and $-1$, $X_{3}$ $\pm1$, and $I$ gives $+1$ and $+1$
		\item In each instance, recovery is done by applying the inverse operation to the error indicated
		\item This is summarised by:
		\begin{table}[h]
			\centering
			\begin{tabular}{l|l|l|l}
				$Z_{1}Z_{2}$ & $Z_{2}Z_{3}$ & Error Type & Action \\ \hline
				+1&       +1&   No error  & Nothing \\
				+1&       -1&   Bit 3 flipped  & Flip bit 3 \\
				-1&       +1&   Bit 1 flipped  &  Flip bit 1\\
				-1&       -1&   Bit 2 flipped  & Flip bit 2
			\end{tabular}
		\end{table}
		\item We don't really gain any insight here - we only see the power on more complex examples
	\end{itemize}
	\section{AdS/CFT (Harlow notes)}
	\begin{itemize}
		\item Reference: https://pos.sissa.it/305/002/pdf
		\item The main statement of AdS/CFT is that `any CFT in $d$-dimensions is equivalent to a theory of quantum gravity in a family of asymptotic $\text{AdS}_{d}\times M$ spacetimes, where $M$ is a compact manifold'
		\item We start with AdS spaces
	\end{itemize}
	\subsection{Anti de Sitter Space}
	\newcommand{\ads}{\ensuremath{AdS_{d+1}}}
	\begin{itemize}
		\item \textit{Anti de Sitter space} is defined as the maximally symmetric spacetime of constant negative curvature
		\item Consider $d+2$-dimensional Minkowski space with $(2,d)$ signature, which has metric
		\begin{equation}
			ds^{2}=-dT_{1}^{2}-dT_{2}^{2}+dX_{1}^{2}+\ldots+dX_{d}^{2}
		\end{equation}
		and define a submanifold via
		\begin{equation}
			T_{1}^{2}+T_{2}^{2}-\mathbf{X}^{2}=\ell^{2}
		\end{equation}
		so $\ell$ has units of length
		\item We can define global coordinates for this submanifold by
		\begin{equation}
			\begin{aligned}
				T_{1}&=\sqrt{\ell+r^{2}}\cos\left(\frac{t}{\ell}\right)\\
				T_{2}&=\sqrt{\ell+r^{2}}\sin\left(\frac{t}{\ell}\right)\\
				\mathbf{X}^{2}&=r^{2}
			\end{aligned}
		\end{equation}
		which we substitute back into (2.1) to obtain the \ads metric in global coordinates as
		\begin{equation}
			ds^{2}=-\left(1+\frac{r^{2}}{\ell^{2}}\right)dt^{2}+\frac{dr^{2}}{1+\frac{r^{2}}{\ell^{2}}}+r^{2}d\Omega_{d-1}^{2}
		\end{equation}
		where $r\in[0,\infty)$ is the usual radial spatial coordinate, $t\in(-\infty,\infty)$ is `time' in some sense, and $\Omega_{d-1}$ is the angular part of the metric for the hypersphere $S^{d-1}$
		\item \ads space has several important properties
		\item Firstly, it solves Einstein's equations
		\begin{equation}
			G_{\mu\nu}\coloneqq R_{\mu\nu}-\frac{1}{2}R g_{\mu\nu}=8\pi G T_{\mu\nu}
		\end{equation}
		with the energy-momentum tensor given by
		\begin{equation}
			T_{\mu\nu}=-\rho_{0}g_{\mu\nu},\qquad \rho_{0}=-\frac{d(d-1)}{16\pi G\ell^{2}}
		\end{equation}
		\item We usually work in units of $\ell$, so $\ell=1$ in what follows, and we can reintroduce through dimensional analysis if needed
		\item A second property is that it has a large symmetry group: $SO(d,2)$, the Lorentz group of $d+2$ dimensions
		\item This group acts \textit{transitively} on \ads (any two points are connected by a symmetry), so \ads is \textit{homogenous}
		\item It is also \textit{locally isotropic}: the subgroup which fixes a chosen $p\in AdS_{d+1}$ can take any $X\in T_{p}(\ads)$ to a multiple of any other tangent vector at $p$
		\item To work out the causal structure of \ads, we define a new coordinate $\rho\in (0,\pi/2)$ by
		\begin{equation}
			r=\tan{\rho}
		\end{equation}
		which transforms the metric (2.4) to
		\begin{equation}
			ds^{2}=\frac{1}{\cos^{2}\rho}(-dt^{2}+d\rho^{2}+\sin^{2}\rho\,d\Omega_{d-1}^{2})
		\end{equation}
		\item Null geodesics by definition satisfy $ds^{2}=0$, so they do not care about the \textit{Weyl factor} (multiplication by a scalar function) of the metric, so the causal structure of \ads is equivalent to the cylinder with metric
		\begin{equation}
			ds^{2}=-dt^{2}+d\rho^{2}+\sin^{2}\rho\,d\Omega_{d-1}^{2}
		\end{equation}
		\item \ads space also has an \textit{asymptotic boundary} at $r=\infty$ (or $\rho=\frac{\pi}{2}$)
		\item This boundary is just the boundary of a hypercylinder, so has topology $\mathbb{R}\times S^{d-1}$
		\item This means that a null geodesic sent from the `center' of the cylinder hits the boundary and returns to its original spatial coordinate in finite proper time
		\item We therefore think of \ads space as behaving in some sense like a `finite box', despite having infinite spatial volume
		\item If matter is present, the global metric (2.4) will be perturbed, so it is natural to define \textit{asymptotically-AdS spacetimes}, which have an asymptotic boundary isomorphic to $\mathbb{R}\times S^{d-1}$ and metric asymptotically approaching (2.4) as we approach said boundary
		\item More generally, we can study spacetimes which approach $\ads\times M$ at the asymptotic boundary, where $M$ is compact with finite volume as $r\to\infty$
		\item \textbf{Important example:} AdS-Schwarzschild geometry
		\begin{itemize}
			\item This has metric
			\begin{equation}
				\begin{aligned}
					ds^{2}&=-f(r)dt^{2}+\frac{dr^{2}}{f(r)}+r^{2}\,d\Omega_{d-1}^{2}\\
					f(r)&=1+r^{2}-\frac{16\pi GM}{(d-1)\Omega_{d-1}}\frac{1}{r^{d-2}}
				\end{aligned}
			\end{equation}
			\item For $d>2$, this gives the \textbf{unique} set of spherically symmetric solutions to Einstein's equations
			\item As $r\to \infty$, this approaches (2.4), but at small $r$, weird things happen
			\item The full geometry at small $r$ describes two asymptotically AdS boundaries connected by a wormhole
		\end{itemize}
		\item A natural thing to do on \ads space is to quantise a free scalar field
		\item This has action
		\begin{equation}
			S=-\frac{1}{2}\int d^{d+1}x\,\sqrt{-g}(\partial_{\mu}\phi\partial_{\nu}\phi g^{\mu\nu}+m^{2}\phi^{2})
		\end{equation}
		where $g=\det{g_{\mu\nu}}$ is the determinant of the metric tensor
		\item This yields equation of motion
		\begin{equation}
			\nabla^{2}\phi=\frac{1}{\sqrt{-g}}\partial_{\mu}\left(\sqrt{-g}g^{\mu\nu}\partial_{\nu}\phi\right)=m^{2}\phi
		\end{equation}
		which has a basis in global coordinates as
		\begin{equation}
			f_{\omega\ell\mathbf{m}}(r,t,\Omega)=\Psi_{\omega\ell}(r)e^{-i\omega t}Y_{\ell\mathbf{m}}(\Omega)
		\end{equation}
		where $\omega^{2}=m^{2}$, $Y_{\ell\mathbf{m}}$ are spherical harmonics, defined to obey
		\begin{equation}
			\nabla^{2}Y_{\ell\mathbf{m}}=-\ell(\ell+d-2)Y_{\ell\mathbf{m}}
		\end{equation}
		for the spherical Laplacian, and $\Psi_{\mathbf{m}\ell}$ obeys
		\begin{equation}
			(1+r^{2})\Psi''+\left(\frac{d-1}{r}(1+r^{2})+2r\right)\Psi'+\left(\frac{\omega^{2}}{1+r^{2}}-\frac{\ell(\ell+d-2)}{r^{2}}-m^{2}\right)\Psi=0
		\end{equation}
		\item At small $r$, this asymptotically becomes
		\begin{equation}
			\Psi''+\frac{d-1}{r}\Psi'-\frac{\ell(\ell+d-2)}{r^{2}}\Psi=0
		\end{equation}
		which has solutions
		\begin{equation}
			\Psi\propto r^{-\frac{d-2}{2}\pm\frac{1}{2}\sqrt{(d-2)^{2}+4\ell(\ell+d-2)}}
		\end{equation}
		\item At large $r$, we instead have
		\begin{equation}
			r^{2}\Psi''+(d+1)r\Psi'-m^{2}\Psi=0
		\end{equation}
		with solutions
		\begin{equation}
			\Psi\propto r^{-\left(\frac{d}{2}\pm\frac{1}{2}\sqrt{d^{2}+4m^{2}}\right)}
		\end{equation}
		\item Smoothness at $r=0$ requires we choose the solution with +
		\item At $r\to\infty$, the sign is specified by boundary conditions
		\item For $m^{2}\geq 0$ and $d\geq 2$, the only choice which maintains unitarity and preserves symmetry is the + sign, so
		\begin{equation}
			\Psi_{\omega\ell}(r)\sim r^{-\Delta},\quad\Delta\coloneqq\frac{d}{2}+\frac{1}{2}\sqrt{d^{2}+4m^{2}}
		\end{equation}
		\item This is known as \textit{standard quantisation}
		\item The full equation (2.15) can be solved by hypergeometric functions, and imposes quantisation of $\omega$ as well - we find
		\begin{equation}
			\omega=\omega_{n\ell}\coloneqq\Delta+\ell+2n
		\end{equation}
		where $n\in\mathbb{N}$
		\item We are now in position to write down the full solution of the scalar field theory in terms of creation/annihilation operators as
		\begin{equation}
			\phi(r,t,\Omega)=\sum_{n,\ell,\mathbf{m}}\left(f_{\omega_{n\ell}\ell\mathbf{m}}a_{n\ell\mathbf{m}}+f_{\omega_{n\ell}\ell\mathbf{m}}^{*}a^{\dagger}_{n\ell\mathbf{m}}\right)
		\end{equation}
		with the creation/annihilation operators obeying
		\begin{equation}
			\left[a_{n\ell\mathbf{m}},a_{n'\ell'\mathbf{m}'}^{\dagger}\right]=\delta_{nn'}\delta_{\ell\ell'}\delta_{mm'}
		\end{equation}
		\item This has the usual particle interpretation from QFT
		\item The quantisation of $\omega$ means that the particle spectrum is discrete - this is \textbf{very} different to Minkowski space, where we have a continuous spectrum of particle energy and momentum
		\item Note the analogy to the QM `particle in a box', which also has a discrete spectrum; this is further evidence that we can think of \ads as being in some sense similar to a finite box
	\end{itemize}
	\subsection{Conformal Field Theories}
	\begin{itemize}
		\item \textit{Conformal field theories (CFTs)} are QFTs invariant under \textit{conformal transformations} (angle-preserving, but not necessarily distance preserving transformations)
		\item More concretely, the conformal group includes translations, Lorentz transformations, and scaling/dilations
		\item This includes \textit{special conformal transformations}, defined as
		\begin{equation}
			x^{\mu\,'}=\frac{x^{\mu}+a^{\mu}x^{2}}{1+2a_{\nu}x^{\nu}+a^{2}x^{2}}
		\end{equation}
		\item The generators of the conformal group $SO(d,2)$ are:
		\begin{equation}
			\begin{aligned}
				D&\coloneqq -ix_{\mu}\partial^{\mu} & \quad &\text{(dilations)}\\
				P_{\mu}&\coloneqq-i\partial_{\mu} & \quad &\text{(translations)}\\
				K_{\mu}&\coloneqq i(x^{2}\partial_{\mu}-2x_{\mu}x_{\nu}\partial^{\nu}) & \quad &\text{(special conformal)}\\
				M_{\mu\nu}&\coloneqq i(x_{\mu}\partial_{\nu}-x_{\nu}\partial_{\mu}) & \quad &\text{(Poincare)}
			\end{aligned}
		\end{equation}
		\item In CFTs we are often interested in \textit{primary operators}, which are local operators which transform as
		\begin{equation}
			\begin{aligned}
				e^{iD\alpha}\mathcal{O}(x)e^{-iD\alpha}&=e^{\alpha\Delta}\mathcal{O}(e^{\alpha}x)\\
				e^{iK_{\mu}\alpha^{\mu}}\mathcal{O}(0)e^{-iK_{\mu}\alpha^{\mu}}&=\mathcal{O}(0)
			\end{aligned}
		\end{equation}
		where $\Delta$ is a scalar known as the \textit{scaling dimension} of $\mathcal{O}$
		\item If we act on a primary operator $\mathcal{O}$ repeatedly by derivatives, we obtain \textit{descendant operators}, which have scaling dimension equal to $\Delta$ plus the number of derivatives
		\item Descendant operators are never primary unless they are identically zero
		\item In a CFT, the set of primary operators and their descendants at any $x$ are in bijection with a complete basis of the Hilbert space of the CFT quantised on $S^{d-1}$, called the \textit{state-operator correspondence}
		\item The map from operators to states is defined by performing a path integral over a Euclidean ball centered on the operator; this is invertible, since if we have a state on the boundary of the ball, we can dilate to shrink the ball down to a point, which defines an operator producing the same state once we scale up again
		\item If the operator has dimension $\Delta$, the state on $S^{d-1}$ has energy $\Delta+E_{0}$, where $E_{0}$ is the ground state energy
		\item Depending on the dimension, we can't always set $E_{0}=0$
		\item This relation comes from choosing polar coordinates near to the Euclidean origin, and then setting $\rho=e^{\tau}$:
		\begin{equation}
			ds^{2}=d\rho^{2}+\rho^{2}d\Omega_{d-1}^{2}=e^{2\tau}(d\tau^{2}+d\Omega_{d-1}^{2})
		\end{equation}
		so dilations $\rho'=e^{\alpha}\rho$ are equivalent to Euclidean cylinder time translations $\tau'=\tau+\alpha$
	\end{itemize}
	\subsection{The Correspondence}
	\begin{itemize}
		\item The AdS/CFT correspondence states: any CFT on $\mathbb{R}\times S^{d-1}$ is equivalent to a theory of quantum gravity in asymptotically $\ads\times M$ spacetime, with $M$ some compact manifold
		\item Immediately, there's two big questions:
		\begin{enumerate}
			\item What is the map between observables on both sides?
			\item Which CFTs define `semiclassical' gravity theories, where the Planck length $\ell_{p}$ is much smaller than the AdS scale $\ell$ and gravity is approximately described by Einstein's equations?
		\end{enumerate}
		\item The solution to Q1 is called the \textit{dictionary}
		\item The simplest way to view the duality is a Hilbert space isomorphism 
		\begin{equation}
			\phi\,:\,\mathcal{H}_{AdS}\to\mathcal{H}_{CFT}
		\end{equation}
		\item We've already seen that the asymptotic symmetry group of \ads and the conformal group in $d$ dimensions are both isomorphic to $SO(d,2)$, so our first line in the dictionary is that the unitary operators implementing these symmetries are related by $\phi$:
		\begin{equation}
			\phi\circ U_{AdS}=U_{CFT}\circ\phi
		\end{equation}
		\item More generally, any operator on one side can be mapped to one on the other using $\phi$
		\item We choose a basis where $\phi$ is just the identity, and assume this from now on
		\item To formulate more of the dictionary, we need to answer Q2: if the bulk space is not semiclassical, we don't know what other observables to study
		\item We therefore say that a $d$ dimensional CFT has a \textit{semiclassical dual near the vacuum} if there exists a finite set of CFT primary operators $\{\mathcal{O}_{i}\}$ and a local bulk effective action $S_{\text{eff}}[\phi_{i},\Lambda]$, where $\Lambda$ is a cutoff which is large compared to $1/\ell$, but is at most $1/\ell_{p}\coloneqq G^{-\frac{1}{d-1}}$, and $\{\phi_{i}\}$ are a finite set of bulk fields including the metric $g$ such that
		\begin{equation}
			\int D\phi_{i}\,e^{iS_{\text{eff}}[\phi_{i},\Lambda]}\mathcal{O}_{i_{1}}(t_{1},\Omega_{1})\ldots\mathcal{O}_{i_{n}}(t_{n},\Omega_{n})\approx\braket{\mathcal{O}_{i_{1}}(t_{1},\Omega_{1})\ldots\mathcal{O}_{i_{n}}(t_{n},\Omega_{n})}_{CFT}
		\end{equation}
		to all orders in $1/\ell\Lambda$, provided $n$ is $O(1)$ in this parameter. The $\mathcal{O}_{i}$ on the LHS are given by the `extrapolate dictionary'
		\begin{equation}
			\lim_{r\to\infty}r^{\Delta_{i}}\phi_{i}(r,t,\Omega)=\mathcal{O}_{i}(t,\Omega)
		\end{equation}
		where $\Delta_{i}$ is the scaling dimension of $\mathcal{O}_{i}$. The bulk field configurations we integrate over obey asymptotically AdS boundary conditions, with the $i\epsilon$ prescription chosen to project onto the vacuum at early and late times, and the CFT expectation values are computed in the ground state on $S^{d-1}$
		\item This definition ensures that `bulk particle physics' arises from the CFT appropriately
		\item (2.31) should be thought of as being analogous to the LSZ reduction formula on flat space
		\item This definition isn't quite enough to produce everything we know about bulk quantum gravity
		\item For example, it says nothing about black holes
		\item We should at least have that the thermal entropy of the CFT on the sphere at high temps. reproduces the black hole entropy we would derive from $S_{\text{eff}}$
		\item A definition that ensures this is as follows: a $d$-dimensional CFT has a \textit{semiclassical dual} if in addition to having a semiclassical dual near the vacuum, (2.30) also holds for more general asymptotically AdS boundary conditions which allow the induced metric on the boundary to be arbitrary but fixed, and the expectation values of the $\mathcal{O}_{i}$s then need to match the CFT correlators on the boundary
	\end{itemize}
	\subsection{Gapped Large-N Theories}
	\begin{itemize}
		\item A natural question to ask is which CFTs have semiclassical duals?
		\item We have some examples from string theory (most notably, asymptotic $AdS_{5}\times S^{5}$), but we want a more general criterion on CFTs which guarantee a semiclassical dual existing
		\item We don't have such a condition yet, but we have one that looks to be sufficient
		\item A \textit{gapped large-N CFT} is a family of CFTs indexed by $N$ such that
		\begin{itemize}
			\item There is a finite set of `single-trace' primaries $\mathcal{O}_{i}$ such that, if they are normalised by $\braket{\mathcal{O}_{i}\mathcal{O}_{i}}\sim N^{0}$, then
			\begin{equation}
				\braket{\mathcal{O}_{i}\mathcal{O}_{j}\mathcal{O}_{k}}\lessapprox\frac{1}{N^{\#/2}}
			\end{equation}
			where $\#$ is $O(N^{0})$
			\item There is only one single-trace primary of spin 2 and scaling dimension $d$, the energy-momentum tensor $T_{\mu\nu}$, and its three point function has a nonzero term which is $O(1/N^{\#/2})$
			\item For any collection of single-trace operators $\{\mathcal{O}_{i_{1}},\ldots,\mathcal{O}_{i_{n}}\}$ with $n\sim N^{0}$, there is an associated `multi-trace' primary $\mathcal{O}_{i_{1}\ldots i_{n}}$ with dimension $\Delta_{i_{1}}+\ldots+\Delta_{i_{n}}$
			\item At leading order in $1/N$, correlators of singla dn multi trace operators can all be computed by Wick contraction, and corrections to this statement are at most order $N^{-\#(n-2)/2}$, where $n$ is the number of single trace operators plus the number of multitrace operators counted including multiplicity
			\item All operators with $\Delta\sim N^{0}$ are single/multi trace primaries and their descendants
		\end{itemize}
		\item We now claim that any gapped large-$N$ theory has a semiclassical dual, and the bulk fields $\phi_{i}$ correspond to the single-trace primaries $\mathcal{O}_{i}$
		\item This has not been proven, but there's significant evidence in favour of it
	\end{itemize}
\end{document}