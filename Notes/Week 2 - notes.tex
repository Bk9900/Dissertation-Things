\documentclass[12pt,a4paper]{article}
\usepackage[utf8]{inputenc}
\usepackage[T1]{fontenc}
\usepackage{amsmath}
\usepackage{amsfonts}
\usepackage{amssymb}
\usepackage{graphicx}
\usepackage{geometry}
\usepackage{mathrsfs}
\usepackage{braket}
\usepackage{simpler-wick}
\usepackage{simplewick}
\usepackage{tikz}
\usepackage{tikz-feynman}
\usepackage{subcaption}
\usepackage{bm}
\usepackage{slashed}
\usepackage{tensor}
\usepackage{hyperref}
\usepackage{mathtools}
%\usepackage{pst-node}
%\usepackage{auto-pst-pdf}
\title{Quantum Error Correction - Notes}
\author{Ben Karsberg}
\date{2021-22}
\newgeometry{vmargin={15mm}, hmargin={20mm,20mm}}
\numberwithin{equation}{section}
\newcommand{\ketbra}[2]{\ket{#1}\bra{#2}}
\newcommand{\ketbras}[1]{\ketbra{#1}{#1}}
\newcommand{\Pc}{P_{\text{code}}}
\begin{document}
	\maketitle
	\section{Week 2: Harlow Toy Model}
	\begin{itemize}
		\item References:
		\begin{itemize}
			\item https://arxiv.org/pdf/1607.03901.pdf (Harlow paper)
			\item https://pos.sissa.it/305/002/pdf (Harlow presentation)
			\item https://arxiv.org/pdf/quant-ph/9901025.pdf\#page=5\&zoom=100,0,0 (Quantum secret sharing)
		\end{itemize}
		\item Harlow presents a toy model of error correction, which we go through
		\item Suppose we wish to protect the qutrit state
		\begin{equation}
			\ket{\psi}=\sum_{i=0}^{2}a_{i}\ket{i}
		\end{equation}
		against erasure
		\item We project to the code subspace spanned by logical qutrits
		\begin{equation}
			\begin{aligned}
				\ket{0_{L}}&=\frac{1}{\sqrt{3}}(\ket{000}+\ket{111}+\ket{222})\\
				\ket{1_{L}}&=\frac{1}{\sqrt{3}}(\ket{012}+\ket{120}+\ket{201})\\
				\ket{2_{L}}&=\frac{1}{\sqrt{3}}(\ket{021}+\ket{102}+\ket{210})
			\end{aligned}
		\end{equation}
		\item There's two points of note here:
		\begin{enumerate}
			\item This code subspace is symmetric under cyclic permutations of the physical qubits
			\item Each individual qutrit is maximally mixed - it is equal parts $\ket{0}$, $\ket{1}$, $\ket{2}$ (this means it functions as a \textit{quantum secret-sharing code} too)
		\end{enumerate}
		\item To actually prepare these logical qutrits, we adjoin two ancillary qutrits in the state
		\begin{equation}
			\ket{\chi}_{23}=\frac{1}{\sqrt{3}}(\ket{00}+\ket{11}+\ket{22})
		\end{equation}
		\item We define a unitary operation $U_{12}$ on the first two qutrits via the permutation
		\begin{equation}
			\begin{aligned}
				\ket{00}&\to\ket{00} & \ket{11}&\to\ket{01} & \ket{22}&\to\ket{02}\\
				\ket{01}&\to\ket{12} & \ket{12}&\to\ket{10} & \ket{20}&\to\ket{11}\\
				\ket{02}&\to\ket{21} & \ket{10}&\to\ket{22} & \ket{21}&\to\ket{20}
			\end{aligned}
		\end{equation}
		\item Then, note that this unitary implements
		\begin{equation}
			\ket{i_{L}}=U_{12}(\ket{i}_{1}\otimes\ket{\chi}_{23})
		\end{equation}
		for all physical basis elements $\ket{i}$
		\item So, the state can be encoded just by doing
		\begin{equation}
			\ket{\psi_{L}}=U_{12}(\ket{\psi}_{1}\otimes\ket{\chi}_{23})
		\end{equation}
		\item We can similarly encode with a unitary on the (1,3) or (2,3) subsystems by symmetry
		\item This explicitly provides the erasure protection procedure: given only two subsystems of $\ket{\psi_{L}}$, we can just apply the corresponding $U_{ij}^{\dagger}$ and we get back the original $\ket{\psi}$ in one of the subsystems
		\item Sticking with Harlow's numbering, suppose the third qutrit is erased, so we recover $\ket{\psi}$ perfectly in the first qutrit
		\item The no-cloning theorem then implies that we should not be able to work out \textbf{any} information about $\ket{\psi}$ given knowledge of just the third qutrit alone
		\item This is quite obvious from (1.6) and from (1.2): the third qutrit in the encoded state is always maximally mixed/proportional to the identity, so contains zero information about the unencoded state on its own as expected
		\item What can we say about the state $\ket{\chi}$?
		\item All we can really say is that $\ket{\chi}$ has to be entangled
		\item Suppose $\ket{\chi}=\ket{\alpha}\otimes\ket{\beta}$, so is in a product state; then, (1.6) becomes
		\begin{equation}
			\ket{\psi_{L}}=U_{12}(\ket{\psi}_{1}\otimes\ket{\alpha}_{2}\otimes\ket{\beta}_{3})
		\end{equation}
		so the third qutrit is sitting in the state $\ket{\beta}$ \textbf{for all} states in the code subspace, so it provides no info about $\ket{\psi}$ which the second qubit didn't already know 
		\item This can also be viewed under the lens of \textit{quantum secret sharing} - Harlow's toy model is a $((2,3))$ threshold scheme, so we should be able to recover full state information from any two qubits, but \textbf{no} information from any one share
		\item $\ket{\chi}$ clearly needs entanglement for the former to hold
		\item This can equivalently be phrased by saying that a $U_{23}$ and $U_{13}$ can't both simultaneously exist if $\ket{\chi}$ is a product state 
		\item This correctability can also be framed in terms of \textit{logical operators}
		\item Consider the operator on a physical qutrit as
		\begin{equation}
			O\ket{i}=\sum_{j}(O)_{ji}\ket{j}
		\end{equation}
		\item We can (in some sense) project this operator up into the code subspace by just defining an operator with the same matrix elements $(O)_{ij}$ acting on the basis for the code subspace:
		\begin{equation}
			O_{L}\ket{i_{L}}=\sum_{j}(O)_{ji}\ket{j_{L}}
		\end{equation}
		\item We can then just arbitrarily define the action of this logical operator on the full 3-qutrit space which the code subspace is a subspace of
		\item Now, consider the case where $O=O_{1}$ acts only on the first qutrit in the toy model
		\item We can define a logical operator on the code subspace which behaves equivalently to $O_{1}$ but which has non-trivial support only on the first and second qutrits:
		\begin{equation}
			O_{12}\coloneqq U_{12}O_{1} U_{12}^{\dagger}
		\end{equation}
		\item Thus any operator acting on a single qutrit can be represented as a logical operator with support on any two of the qutrits
		\item Harlow says that this is an analogue of \textit{subregion duality} in AdS/CFT
		\item He also says it is an analogue of \textit{radial commutativity}, meaning that any logical operator $O_{L}$ on $\mathcal{H}_{\text{code}}$ commutes with any operator acting on a single qutrit
		\item But $O_{12}$ clearly commutes with any operator $X_{3}$ acting on qutrit 3, and similar for $O_{13}$ and $O_{23}$; since all of these act identically to $O_{L}$ on $\mathcal{H}_{\text{code}}$, it must be that $O_{L}$ commutes with all single qutrit operators
		\item More concretely, for any two states $\ket{\psi_{L}}$ and $\ket{\phi_{L}}$ in $\mathcal{H}_{\text{code}}$, we have
		\begin{equation}
			\braket{\psi_{L}|[O_{L},X]|\phi_{L}}=0
		\end{equation}
		where $X$ is any single qutrit operator
	\end{itemize}
\end{document}