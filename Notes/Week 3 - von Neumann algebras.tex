\documentclass[12pt,a4paper]{article}
\usepackage[utf8]{inputenc}
\usepackage[T1]{fontenc}
\usepackage{amsmath}
\usepackage{amsfonts}
\usepackage{amssymb}
\usepackage{amsthm}
\usepackage{graphicx}
\usepackage{geometry}
\usepackage{mathrsfs}
\usepackage{braket}
\usepackage{simpler-wick}
\usepackage{simplewick}
\usepackage{tikz}
\usetikzlibrary{quantikz}
\usepackage{tikz-feynman}
\usepackage{subcaption}
\usepackage{bm}
\usepackage{slashed}
\usepackage{tensor}
\usepackage{hyperref}
\usepackage{mathtools}
\usepackage{float}
\usepackage{nicematrix}
%\usepackage{pst-node}
%\usepackage{auto-pst-pdf}
\title{Quantum Error Correction - Notes}
\author{Ben Karsberg}
\date{2021-22}
\newgeometry{vmargin={15mm}, hmargin={20mm,20mm}}
\numberwithin{equation}{section}
\newcommand{\ketbra}[2]{\ket{#1}\bra{#2}}
\newcommand{\ketbras}[1]{\ketbra{#1}{#1}}
\newcommand{\Pc}{P_{\text{code}}}
\newcommand{\Hcode}{\mathcal{H}_{\text{code}}}
\newcommand{\ntr}{\hat{\text{Tr}}}
\theoremstyle{definition}
\newtheorem{definition}{Definition}[section]
\theoremstyle{theorem}
\newtheorem{theorem}{Theorem}[section]
\newtheorem{proposition}{Proposition}
\begin{document}
	\maketitle
	\section{von Neumann Algebras}
	\newcommand{\lops}{\mathcal{L}(\mathcal{H})}
	\begin{itemize}
		\item At some point, we will need the theory of \textit{von Neumann algebras}, finite dimensional ones only
		\item Let's go through this
		\item Suppose $\mathcal{H}$ is a finite-dimensional Hilbert space, and denote the set of linear operators on $\mathcal{H}$ as $\lops$
		\item We denote the identity on $\mathcal{H}$ as $I\in\lops$
		\begin{definition}[von Neumann algebra]
			A \textbf{von Neumann algebra} on $\mathcal{H}$ is a set $M\subseteq\lops$ such that:
			\begin{itemize}
				\item $\forall \lambda\in\mathbb{C},\;\lambda I\in M$
				\item $\forall x\in M,\,x^{\dagger}\in M$
				\item $\forall x,\,y\in M,\; xy\in M$
				\item $\forall x,\,y\in M,\; x+y\in M$
			\end{itemize}
		\end{definition}
		\item Essentially: a von Neumann algebra on $\mathcal{H}$ is a set of linear operators which is closed under Hermitian conjugation, addition, multiplication, and contains all scalar multiples of the identity
		\item We note that this definition is only true for \textit{finite dimensional} von Neumann algebras
		\item The `true' definition is topological, but reduces to this above definition when $\mathcal{H}$ is finite dimensional, and this is the only case we need
		\item We often define a von Neumann algebra through its generators, writing $M=\braket{a,b,\ldots}_{vN}$
		\item For example, if $x,y,z$ denote the Pauli matrices, $\braket{z}_{vN}$ is the algebra of $2\times 2$ diagonal matrices, $\braket{z,x}_{vN}=\mathcal{L}(\mathbb{C}^{2})$
		\item Any von Neumann algebra induces two `natural' associated algebras:
		\begin{definition}[Commutant]
			Given von Neumann algebra $M$ on $\mathcal{H}$, the \textbf{commutant} of $M$, denoted $M'$, is
			\begin{equation}
				M'\equiv\left\{y\in\lops\,|\,xy=yx,\,\forall x\in M\right\}
			\end{equation}
		\end{definition}
		\begin{definition}[Center]
			Given von Neumann algebra $M$ on $\mathcal{H}$, the \textbf{center} of $M$, denoted $Z_{M}$, is
			\begin{equation}
				Z_{M}\equiv M\cap M'
			\end{equation}
		\end{definition}
		\item Basically, the commutant $M'$ is the set of all linear operators which commute with the von Neumann algebra $M$, and the center $Z_{M}$ is the subset of these which are themselves in $M$
		\item These are easily checked to be von Neumann algebras themselves
	\end{itemize}
	\subsection{Projections and Partial Isometries}
	\begin{itemize}
		\item Two recurrent classes of linear operators are the projections and partial isometries
		\begin{definition}[Projection]
			A linear map $p\in\lops$ is called a \textbf{projection} if $p^{\dagger}=p$ and $p^{2}=p$
		\end{definition}
		\begin{definition}[Partial Isometry]
			A linear map $a\in\lops$ is called a \textbf{partial isometry} if $a^{\dagger}a=p$, where $P$ is a projection
		\end{definition}
		\item Projections meet our usual intuition of projections, and partial isometries are isometries (distance preserving transformations) on the orthogonal complement to their kernel (sometimes called the \textit{initial subspace})
		\item Projections always have a subspace $p\mathcal{H}$ on which they act identically, and they annihilate the orthogonal complement $(1-p)\mathcal{H}$ (i.e. $\ker{p}=(1-p)\mathcal{H}$)
		\item Partial isometries are characterised by the following theorem:
		\begin{theorem}
			Suppose $a\in\lops$ is a partial isometry, so $a^{\dagger}a=p$ for some projection $p\in\lops$. Then, $a^{\dagger}$ is also a partial isometry, obeying $aa^{\dagger}=q$ where $q\in\lops$ is also a projection, and there exists a unitary $u\in\lops$ such that $q=upu^{\dagger}$. This means $p$ and $q$ have equal rank, and we can in fact choose $u$ such that $a=up$.
		\end{theorem}
		\begin{proof}
			We first show $a^{\dagger}$ is indeed a partial isometry. First, note that any $\ket{v}\in (1-p)\mathcal{H}$ is also annihilated by $a$, since 
			\begin{equation} 
				\lVert a\ket{v}\rVert^{2} =\braket{v|a^{\dagger}a|v}=\braket{v|p|v}=0 \implies a\ket{v}=0
			\end{equation} 
			We can represent $a$ in block-matrix form according to the direct sum representation $\mathcal{H}=p\mathcal{H}\oplus(1-p)\mathcal{H}$ as
			\begin{equation}
				a=\begin{pNiceMatrix}[columns-width=4mm]
					\Block{2-2}<\large>{A} && \Block{2-2}<\large>{\boldsymbol{0}} &\\
					&&&\\
					\Block{2-2}<\large>{B} && \Block{2-2}<\large>{\boldsymbol{0}} &\\
					&&&
				\end{pNiceMatrix}
			\end{equation}
			where only the first column can be non-zero. Computing $a^{\dagger}a=p$, the upper-left block must act identically on $p\mathcal{H}$, so we find $A^{\dagger}A+B^{\dagger}B=I_{p\mathcal{H}}$. From this block representation, we can easily compute $(aa^{\dagger})^{2}=aa^{\dagger}$, so $a^{\dagger}$ is a partial isometry and $q=aa^{\dagger}$ is a projection.\\
			To see that $p$ and $q$ have equal rank, we first note that for any $\ket{v}\in p\mathcal{H}$:
			\begin{equation}
				qa\ket{v}=aa^{\dagger}a\ket{v}=ap\ket{v}=a\ket{v}\implies \ket{v}\in q\mathcal{H}
			\end{equation}
			We also find that for all $\ket{v_{1}},\ket{v_{2}}\in p\mathcal{H}$, we have:
			\begin{equation}
				\braket{v_{1}|a^{\dagger}a|v_{2}}=\braket{v_{1}|p|v_{2}}=\braket{v_{1}|v_{2}}
			\end{equation}
			So, if we choose $\ket{v_{i}}$ to be an orthonormal basis of $p\mathcal{H}$, then the vectors $a\ket{v_{i}}$ are orthonormal too and are in $q\mathcal{H}$. Therefore $\dim{p\mathcal{H}}\leq\dim{q\mathcal{H}}$. Repeating this argument for $q\mathcal{H}$ and $a^{\dagger}$, we find too that $\dim{p\mathcal{H}}\leq\dim{q\mathcal{H}}$, so $\dim{p\mathcal{H}}=\dim{q\mathcal{H}}$, and hence $p$ and $q$ must have equal rank.\\
			Any two projections of equal rank are always unitarily equivalent, so we therefore must have $q=upu^{\dagger}$ for a unitary $u\in\lops$. We can also choose $u$ such that $u\ket{v}=a\ket{v}$ for any $\ket{v}\in p\mathcal{H}$ (\textbf{why??}), which gives us $a=up$. 
		\end{proof}
		\item If projections $p$ and $q$ are related by some partial isometry $a$ in this way, we say $p$ and $q$ are \textit{equivalent}, and write $p\sim q$ (this is indeed an equivalence relation between projections)
		\item Partial isometries also pop up in the polar decomposition:
		\begin{theorem}[Polar Decomposition]
			Suppose $x\in\lops$. Then there exists non-negative matrix $|x|$ and partial isometry $a$ such that $x=a|x|$, where $a^{\dagger}a=p$ projects onto the orthogonal complement $\ker{x}^{\perp}$. Moreover, $a$ and $|x|$ are both unique.
		\end{theorem}
		\begin{proof}
			First, define $|x|\equiv\sqrt{x^{\dagger}x}$ (i.e. $|x|^{2}=x^{\dagger}x$), which is clearly non-negative. Note that 
			\begin{equation} 
				|x|\ket{v}=0\iff\braket{v||x|^{2}|v}=\braket{v|x^{\dagger}x|v}=0\iff x\ket{v}=0
			\end{equation}
			for $\ket{v}\in\mathcal{H}$, so $\ker{x}=\ker{|x|}$. Now, $|x|$ is invertible on $\ker{|x|}^{\perp}=\ker{x}^{\perp}\equiv p\mathcal{H}$ for some projection onto this subspace; so define $a\equiv x\left(|x|^{-1}\oplus 0_{\ker{x}}\right)$, and compute
			\begin{equation}
				a^{\dagger}a=\left(|x|^{-1}\oplus 0_{\ker{x}}\right)|x|^{2}\left(|x|^{-1}\oplus 0_{\ker{x}}\right)=I_{p\mathcal{H}}\oplus 0_{(1-p)\mathcal{H}}=p
			\end{equation}
			Note that $x=a|x|$ implies $x^{\dagger}x=|x|a^{\dagger}a|x|=|x|p|x|=|x|^{2}$, so $|x|$ is unique. Also, if $a'|x|=a|x|$, if we right-multiply by $\left(|x|^{-1}\oplus 0_{\ker{x}}\right)$ we find $a=a'$, so $a$ is also unique. 
		\end{proof}
	\end{itemize}
	\subsection{The Bicommutant Theorem}
	\begin{itemize}
		\item This is arguably the single most important result about von Neumann algebras
		\begin{theorem}
			For any von Neumann algebra $M$ on $\mathcal{H}$, we have $M''\equiv(M')'=M$
		\end{theorem}
		\item Before proving this, note the subtlety: $M''\supseteq M$ by definition, but $M''\subseteq M$ isn't immediate as there may be operators commuting with everything in $M'$ which are outside $M$, so this is what we need to show
		\begin{proof}
			This proof relies on a `doubling' trick. Instead of considering the action of $M$ on $\mathcal{H}$, we extend $M$ to a new von Neumann algebra $I\otimes M$ on $\mathcal{H}\otimes \mathcal{H}$.\\
			Denote $\dim{\mathcal{H}}=n$; then elements of $\mathcal{L}(\mathcal{H}\otimes\mathcal{H})$ can be viewed as $n\times n$ block matrices, where each block is itself $n\times n$. Elements of $I\otimes M$ are block diagonal in this representation, with each diagonal block being the same $x\in M$:
			\begin{equation}
				I\otimes x=\begin{pmatrix}
					x&0&\dots&0\\
					0&\ddots&&\vdots\\
					\vdots&&\ddots&0\\
					0&\dots&0&x
				\end{pmatrix}
			\end{equation}
			In other words, we are doing a block decomposition based on the fact that $\mathcal{H}\otimes\mathcal{H}\cong \oplus_{i=1}^{n}\mathcal{H}_{i}$. To find the commutant, consider an arbitrary element $y\in\mathcal{L}(\mathcal{H}\otimes \mathcal{H})$, where each block is an arbitrary element $y_{ij}\in \lops$. Left and right-multiplying by the matrix $I\otimes x$ shows that in order for $y\in(I\otimes M)'$, we need each individual block $y_{ij}$ to commute with $x$, so since $x$ is arbitrary we must have $y_{ij}\in M'$. This determines the commutant to be the set of all block matrices with all blocks being arbitrary elements of $M'$.\\
			Now consider those elements of $(I\otimes M)'$ where all blocks are zero except for one, which is taken to be the identity. Any element of $(I\otimes M)''$ must certainly commute with all these elements, and some matrix multiplication confirms that such an operator much have the same element $z\in M''$ on each diagonal block with all other blocks zero. This defines the bicommutant $(I\otimes M)''$.\\
			Now, consider an arbitrary vector $\ket{v}\in \mathcal{H}\otimes\mathcal{H}$, and the subspace $V\subseteq\mathcal{H}\otimes\mathcal{H}$ defined by $V\equiv(I\otimes M)\ket{v}$. We claim that the projection $p_{V}$ onto $V$ commutes with all elements of $I\otimes M$. To see this, first note that $V$ is invariant under the action of $I\otimes M$. This implies that for all $x\in I\otimes M$, $p_{V}xp_{V}=xp_{V}$. Fix such an $x$; since $x^{\dagger}\in I\otimes M$, we have $p_{V}x^{\dagger}p_{V}=x^{\dagger}p_{V}$ too, so taking adjoints we find that $p_{V}x=xp_{V}$ for all $x\in I\otimes M$.\\
			Therefore $p_{V}$ commutes with everything in $(I\otimes M)''$, which implies that any element of $(I\otimes M)''$ with $z\in M''$ on the diagonal blocks must also preserve $V$. This means that its action on $\ket{v}$ must be equivalent to the action of some element of $I\otimes M$ with $x\in M$ on the diagonal blocks. Now if we choose a basis $\ket{v_{i}}$ of $\mathcal{H}$ and set $\ket{v}=\oplus_{i}\ket{v_{i}}$, we find that $z=x$ and hence that $M''\subseteq M$. Since $M\subseteq M''$ by definition, we must have $M''=M$.
		\end{proof}
		\item Note that we \textbf{need} the identity axiom for this proof to be valid, else we may not have $\ket{v}\in V$ and so we could not conclude that $z\ket{v}\in V$
	\end{itemize}
	\subsection{Properties of von Neumann Algebras}
	\begin{itemize}
		\item We now state and prove some basic properties of von Neumann algebras
		\begin{proposition}
			Suppose $x\in M$ is Hermitian. Then, the projections onto the eigenspaces of $x$ are also in $M$. Also, if $f\,:\,D\to \mathbb{C}$ with $D\subseteq \mathbb{R}$ is an arbitrary function, and all eigenvalues of $x$ are in $D$, then $f(x)\in M$.
		\end{proposition}
		\begin{proof}
			Since any $y\in M'$ commutes with $x$, we must have that all eigenprojectors commute with all such $y$. This means that the projectors are in $M''= M$ by the bicommutant theorem. Once we have all the projectors $p_{i}$ corresponding to eigenvalues $\lambda_{i}$, we can decompose $x=\sum_{i}\lambda_{i}p_{i}$ which allows us to define $f(x)=\sum_{i}f(\lambda_{i})p_{i}$, which is clearly in $M$. 
		\end{proof}
		\begin{proposition}
			Any $x\in M$ can be written as a linear combination of at most 4 unitary elements of $M$.
		\end{proposition}
		\begin{proof}
			First, note that $x\in M$ can be written as a linear combination of two Hermitian elements of $M$ via
			\begin{equation}
				x=\frac{x+x^{\dagger}}{2}+i\frac{x-x^{\dagger}}{2i}
			\end{equation}
			so we just need to show that Hermitian operators can be written as a linear combination of two unitaries. Suppose then that $x^{\dagger}=x$. We can rescale $x$ so that its largest eigenvalue has magnitude less than 1, in which case we can write
			\begin{equation}
				x=\frac{1}{2}\left(x+i\sqrt{1-x^{2}}\right)+\frac{1}{2}\left(x-i\sqrt{1-x^{2}}\right)
			\end{equation}
			The operators $x\pm i\sqrt{1-x^{2}}$ are unitary, and by prop. 1 they are in $M$ too. 
		\end{proof}
		\begin{proposition}
			Suppose $p\in M$ is a projection. Then $pMp$ defines a von Neumann algebra on $p\mathcal{H}$, and its commutant on $p\mathcal{H}$ is $M'p$.
		\end{proposition}
		\begin{proof}
			$pMp$ can be easily (but tediously) checked to be a von Neumann algebra satisfying the axioms. To show that $M'p$ is the commutant, we can abuse the bicommutant theorem and just show that $pMp=(M'p)'$. Suppose that $x\in \mathcal{L}(p\mathcal{H})$ commutes with $yp$ for all $y\in M'$. If we define $x_{0}\equiv x\oplus 0_{(1-p)\mathcal{H}}$, then clearly $x=px_{0}p$. To check that $x_{0}\in M$, we again use the bicommutant theorem - if $x_{0}$ commutes with all $y\in M'$, then it will be in $M''=M$. But $x_{0}y=x_{0}py=x_{0}yp=ypx_{0}=yx_{0}$, so we are done. 
		\end{proof}
		\begin{proposition}
			Suppose $x\in M$, and that it has unique polar decomposition $x=a|x|$. Then $a$ and $|x|$ are both in $M$ too.
		\end{proposition}
		\begin{proof}
			$|x|=\sqrt{x^{\dagger}x}\in M$ by prop. 1. To show $a\in M$, we show that it commutes with $M'$, and hence is in $M$ by the bicommutant theorem. By prop. 2, it suffices to show $a$ commutes with any unitary element $u\in M'$. First, note that $u(a|x|)=(a|x|)u=au|x|$ since $x=a|x|$ and $|x|$ are both in $M$. But then by theorem 1.2, the projection $a^{\dagger}a$ onto the orthogonal complement $\ker{x}^{\perp}=\ker{|x|}^{\perp}$ is also in $M$. This therefore means that $(au)^{\dagger}(au)=u^{\dagger}(a^{\dagger}a)u=a^{\dagger}au^{\dagger}u=a^{\dagger}a=a^{\dagger}u^{\dagger}ua=(ua)^{\dagger}(ua)$, so by uniqueness of the polar decomposition of $ua|x|$, we have $ua=au$. 
		\end{proof}
	\end{itemize}
	\subsection{Factors}
	\begin{itemize} 
		\item A special type of von Neumann algebras are \textit{factors}
		\begin{definition}[Factor]
			A von Neumann algebra $M$ on $\mathcal{H}$ is called a \textbf{factor} if its center $Z_{M}\equiv M\cap M'$ contains only scalar multiples of the identity $I$.
		\end{definition}
		\item Factors have a special role, and they have some nice properties:
		\begin{proposition}
			Suppose $M$ is a factor, and $p$ and $q$ are nonzero projectors in $M$. Then there exists a unitary $u\in M$ such that $puq\neq 0$.
		\end{proposition}
		\begin{proof}
			The proof is by contradiction. Suppose that $puq=0$ for all unitaries $u\in M$. Then we also have that $u^{\dagger}puq=0$ for all unitaries. So define a new projection $r$ which annihilates only those vectors in $\cap_{u\in M}\ker{u^{\dagger}pu}$ (i.e. $\ker{r}=\cap_{u\in M}\ker{u^{\dagger}pu}$). We note two things:
			\begin{itemize}
				\item Any vector in $q\mathcal{H}$ is annihilated by $u^{\dagger}pu$, and so is also annihilated by $r$, so $r$ is not the identity.
				\item Since $p\neq0\implies u^{\dagger}pu\neq0$, $r$ is non-zero too.
			\end{itemize}
			Also note that $\ker{r}$ is preserved by the action of any unitary $\hat{u}$. To see this, suppose $\ket{v}\in \cap_{u\in M}\ker{u^{\dagger}pu}$; then:
			\begin{equation}
				u^{\dagger}pu\ket{v}=0\;\forall u\in M\implies \hat{u}u^{\dagger}pu\hat{u}^{\dagger}\hat{u}\ket{v}=0\;\forall u\in M
			\end{equation}
			But if we have a complete set of unitaries $\{u\}$, then the set $\{u\hat{u}\}$ for fixed $\hat{u}$ is just a relabelling and is also the same complete set of unitaries (\textbf{can I prove this? Is there a neater way of expressing this argument?}). Therefore $\hat{u}$ preserves $\ker{r}$ as claimed. But this means $r$ commutes with all unitaries $\hat{u}$ (since $\hat{u}$ acts within $\ker{r}$, and $r$ is the identity on $\text{dom}\,{r}$), and thus with everything in $M$. $r$ is itself also in $M$ since it commutes with everything in $M'$. To see this, note that for fixed $x\in M'$ and arbitrary $\ket{v}\in\cap_{u\in M}\ker{u^{\dagger}pu}$, $xu^{\dagger}pu\ket{v}=u^{\dagger}pux\ket{v}=0$ for all $u$, since $x$ commutes with $u^{\dagger}pu\in M$. Therefore, $rx\ket{v}=xr\ket{v}=0$, and so $r$ commutes with $x$. $r$ is therefore a non-trivial element of $Z_{M}$, which contradicts $M$ being a factor. 
		\end{proof}
		\item Before presenting the next property, we introduce a (partial) ordering on projections
		\item For projections $p$ and $q$, if $p\mathcal{H}\subseteq q\mathcal{H}$ (or equivalently, $\ker{p}\supseteq\ker{q}$), we say $p\leq q$
		\item Note in particular that if $p\mathcal{H}\perp q\mathcal{H}$, then we can't compare $p$ and $q$ with this ordering
		\begin{proposition}
			Suppose $M$ is a factor, and $p$ and $q$ are non-zero projectors in $M$. Then, there exists a partial isometry $a$ such that $a^{\dagger}a\leq q$ and $aa^{\dagger}\leq p$.
		\end{proposition}
		\begin{proof}
			Define $x\equiv puq$, with $u\in M$ a unitary chosen so $x\neq 0$. By the polar decomposition, we have $x=a|x|$, and note that $\ker{a}=\ker{|x|}$. If $\ket{v}\in \ker{q}$, then $puq\ket{v}=x\ket{v}=a|x|\ket{v}=0$, so $\ket{v}\in \ker{|x|}=\ker{a}$, which means $a^{\dagger}a\leq q$. Also, $qu^{\dagger}p=|x|a^{\dagger}$, if $\ket{v}$ is annihilated by $p$ it must also be annihilated by $|x|a^{\dagger}$. From thm. 1.1, we know that $a^{\dagger}=a^{\dagger}aw^{\dagger}$, where $w$ is a unitary that maps $\ker{|x|}$ to $\ker{a^{\dagger}}$, so $|x|a^{\dagger}\ket{v}=0\implies a^{\dagger}\ket{v}=0$, so $aa^{\dagger}\leq p$.
		\end{proof}
		\item We now introduce a special type of projection
		\begin{definition}[Minimal Projection]
			Suppose $M$ is a von Neumann algebra on $\mathcal{H}$, and $p$ is a non-zero projection. We say $p$ is a \textbf{minimal projection} if for any projection $q\in M$, we have $q\leq p$ iff $q=0$ or $q=p$.
		\end{definition}
		\item Since $\mathcal{H}$ is finite dimensional, minimal projections must always exist in any von Neumann algebra
		\item To see this, suppose we have a non-zero, non-minimal projection $p$
		\item We can find a non-zero projection $q$ of smaller rank such that $q\leq p$
		\item If $q$ is non-minimal, we can repeat this, and since any projection of rank 1 is necessarily minimal, this procedure always finds a minimal projection
		\item Minimal projections are characterised by the following:
		\begin{theorem}
			Suppose $M$ is a von Neumann algebra on $\mathcal{H}$, and $p$ is a minimal projection. Then $pMp=\mathbb{C}p$.
		\end{theorem}
		\begin{proof}
			$pMp$ always contains $\mathbb{C}p$ trivially. If it contains any other operators, then by prop. 1 it has a non-trivial projection $q$. But such a $q$ contradicts $p$ being minimal. 
		\end{proof}
		\item Minimal projections existing is a consequence of $\mathcal{H}$ being finite dimensional
		\item In the infinite dimensional case, factors containing a minimal projection are called \textit{type I}, while those that don't are called \textit{type II/III}
	\end{itemize}
	\subsection{Classification of Finite Dimensional von Neumann Algebras}
	\begin{itemize}
		\item We are now in a position to classify all von Neumann algebras on finite dimensional Hilbert spaces
		\item The trickiest step is classifying factors - we start with this
		\begin{theorem}[Factor Classification]
			Suppose $M$ is a factor on $\mathcal{H}$. Then there exists a tensor factorisation $\mathcal{H}=\mathcal{H}\otimes\mathcal{\overline{A}}$ such that $M=\mathcal{L}(\mathcal{H}_{A})\otimes I_{\overline{A}}$, and moreover $M'=I_{A}\otimes\mathcal{L}_{\mathcal{H}_{\overline{A}}}$
		\end{theorem}
		\begin{proof}
			Let $\{p_{1},p_{2},\ldots\}$ be a maximal set of minimal projections, satisfying $p_{i}p_{j}=0$ for $i\neq j$. Such a set always exists since we can take any single minimal projection and then keep adding more until we can no longer do so. Our first claim is that $\sum_{i}p_{i}=I$. Define
			\begin{equation}
				q=1-\sum_{i}p_{i}
			\end{equation}
			and note that $q$ is itself a projector, with $qp_{i}=p_{i}q=0$ for all $i$. Assume $q\neq 0$. Since our set is maximal, $q$ is not itself minimal, but we can generate a minimal projector by finding a non-zero projection $\hat{q}$ of smaller rank with $\hat{q}\leq q$ as described above, repeating this process until we have a new minimal projector. $\hat{q}$ is not in our set of $p_{i}$'s since its domain is mutually orthogonal to them all, which contradicts maximality. Hence $q=0$, and $I=\sum_{i}p_{i}$.\\
			For each $i$, we have a partial isometry $a_{i}$ satisfying $a_{i}^{\dagger}a_{i}\leq p_{i}$ and $a_{i}a_{i}^{\dagger}\leq p_{1}$. By minimality, we must have $a_{i}^{\dagger}a_{i}=p_{i}$ and $a_{i}a_{i}^{\dagger}=p_{1}$. However, theorem 1.1 tells us that all the projectors are unitarily equivalent, and so have equal rank. Since $I=\sum_{i}p_{i}$ has full rank equal to $\dim{\mathcal{H}}$, this rank must divide $\dim{\mathcal{H}}$.\\
			Moreover, since $I=\sum_{i}p_{i}$, we have $x=\sum_{i,j}p_{i}xp_{j}$ for any $x\in M$. We now note 
			\begin{equation}
				p_{i}xp_{j}=p_{i}^{2}xp_{j}^{2}=a_{i}^{\dagger}a_{i}a_{i}^{\dagger}a_{i}xa_{j}^{\dagger}a_{j}a_{j}^{\dagger}a_{j}=a_{i}^{\dagger}p_{1}a_{i}xa_{j}^{\dagger}p_{1}a_{i}
			\end{equation}
			Since $p_{1}$ is minimal, we have by theorem 1.4 that there exist complex coefficients $\lambda_{ij}\in\mathbb{C}$ such that $p_{1}a_{i}xa_{j}^{\dagger}p_{1}=\lambda_{ij}p_{1}$. Recalling that $a_{i}$ maps $p_{i}\mathcal{H}\to p_{1}\mathcal{H}$, we have:
			\begin{equation}
				p_{i}xp_{j}=\lambda_{ij}a_{i}^{\dagger}p_{1}a_{j}=\lambda_{ij}a_{i}^{\dagger}a_{j}
			\end{equation}
			and so finally, $x=\sum_{ij}\lambda_{ij}a_{i}^{\dagger}a_{j}$. This means that $M$ is generated by the $a_{i}$'s.\\
			We now identify the algebra generated like this. Since $I=\sum_{i}p_{i}$, we have $\mathcal{H}=\oplus_{i}p_{i}\mathcal{H}$. We can therefore define a tensor product structure $\mathcal{H}=\mathcal{H}_{A}\otimes\mathcal{H}_{\overline{A}}$ by taking $\mathcal{L}(\mathcal{H}_{A})\otimes I_{\overline{A}}$ to be the block matrices where each block is an arbitrary multiple of the identity, and $I_{A}\otimes\mathcal{L}(\mathcal{H}_{\overline{A}})$ to be the set of block diagonal matrices with the same element of $\mathcal{L}(\mathcal{H}_{\overline{A}})$ in each diagonal block. We can choose a basis within each block so that $a_{i}$ is represented as a block matrix with the identity in the $(1,i)$th block, with zeros elsewhere. Then, $a_{i}^{\dagger}a_{j}$ has the identity in the $(i,j)$th block with zeros elsewhere. These matrices clearly generate $\mathcal{L}(\mathcal{H}_{A})\otimes I_{\overline{A}}$, which is thus $M$. It's not hard to verify that $M'=I_{A}\otimes\mathcal{H}(\mathcal{L}_{\overline{A}})$. 
		\end{proof}
		\item Note that this theorem justifies why they're called `factors' - if a Hilbert space admits a factor, it factorises
		\item Let's now consider the general case where $M$ is not necessarily a factor
		\item The basic logic is that since all elements of the center $Z_{M}$ mutually commute, we can simultaneously diagonalise them in some basis
		\item By prop. 1, this means there is a family of mutually orthogonal projections $p_{\alpha}\in Z_{M}$ such that $Z_{M}$ is equivalent to the set of operators $\sum_{\alpha}\lambda_{\alpha}p_{\alpha}$
		\item We then have the following proposition:
		\begin{proposition}
			Suppose $M$ is a von Neumann algebra, with center $Z_{M}$ spanned by mutually orthogonal projections $p_{\alpha}$. Then for all $\alpha$, $p_{\alpha}Mp_{\alpha}$ is a factor on $p_{\alpha}\mathcal{H}$. Also, if $\alpha\neq \beta$, then $p_{\alpha}Mp_{\beta}=0$.
		\end{proposition}
		\begin{proof}
			Suppose $p_{\alpha}Mp_{\alpha}$ had a non-trivial central element $c$. Then, $c\oplus 0_{(1-p_{\alpha})\mathcal{H}}$ is not in the span of the $p_{\alpha}$s, but they span $Z_{M}$ so no such $c$ can exist, and so $p_{\alpha}Mp_{\alpha}$ is a factor. Moreover, if $\alpha\neq\beta$, then $p_{\alpha}Mp_{\beta}=Mp_{\alpha}p_{\beta}=0$.
		\end{proof}
		\item What this proposition basically says is that if we decompose $\mathcal{H}=\oplus_{\alpha}p_{\alpha}\mathcal{H}$, then every element of $M$ is block diagonal, and each diagonal block is itself a factor algebra
		\item Together with theorem 1.5, this implies the full classification of von Neumann algebras:
		\begin{theorem}[Classification of von Neumann Algebras]
			Suppose $M$ is a von Neumann algebra on finite dimensional Hilbert space $\mathcal{H}$. Then, we have a block decomposition $\mathcal{H}=\oplus_{\alpha}\left(\mathcal{H}_{A_{\alpha}}\otimes\mathcal{H}_{\overline{A}_{\alpha}}\right)$ in terms of which $M$ and $M'$ are block diagonal, with decompositions $M=\oplus_{\alpha}\left(\mathcal{L}(\mathcal{H}_{A_{\alpha}})\otimes I_{\overline{A}_{\alpha}}\right)$ and $M'=\oplus_{\alpha}\left(I_{A_{\alpha}}\otimes\mathcal{L}(\mathcal{H}_{\overline{A}_{\alpha}})\right)$.
		\end{theorem}
		\item Note the abuse of notation: if we have a block diagonal operator $x$ with diagonal blocks $x_{\alpha}$, then we can just write $x=\oplus_{\alpha}x_{\alpha}$
		\item This theorem actually has an infinite dimensional analogue, but classifying factors is much trickier so we ignore it for now - infinite dimensional von Neumann algebras only really come up in QFT, and quantum computing is strictly quantum mechanical
	\end{itemize}
	\subsection{Entropy}
	\subsubsection{States}
	\begin{itemize}
		\item So far, we've looked at von Neumann algebras as subsets of $\lops$
		\item In QM, hermitian operators correspond to observables, but we need to introduce states in order to do physics
		\begin{definition}[States]
			A linear operator $\rho\in\lops$ is called a \textbf{state} on $\lops$ if it is hermitian, non-negative, and has $\text{Tr}(\rho)=1$.
		\end{definition}
		\item Any state $\rho$ has a canonical linear action $\mathbb{E}_{\rho}$ on $\lops$
		\begin{definition}[Expectation]
			For any hermitian $x\in\lops$, the \textbf{expectation value of} $x$ \textbf{in the state} $\rho$ is
			\begin{equation}
				\mathbb{E}_{\rho}(x)=\text{Tr}(\rho x)
			\end{equation}
		\end{definition}
		\item This can more generally be defined as a linear action on any $x\in\lops$
		\item In more mathematical presentations of this, states are often \textit{defined} as linear, non-negative maps on $\lops$ obeying $\mathbb{E}_{\rho}(I)=1$, but this is needlessly abstract for us
		\item Sometimes, one is interested only in observables which are elements of a von Neumann algebra $M$
		\item A generic state $\rho$ will not necessarily be in $M$, and will often contain more information than necessary to compute expectation values of elements in $M$
		\item The following theorem gives a way to discard this additional information:
		\begin{theorem}
			Suppose $M$ is a von Neumann algebra on $\mathcal{H}$, and $\rho$ is a state on $\mathcal{H}$. Then, there exists a unique state $\rho_{M}\in M$ such that $\mathbb{E}_{\rho}(x)=\mathbb{E}_{\rho_M}(x)$ for all $x\in M$.
		\end{theorem}
		\begin{proof}
			The basic point is to define
			\begin{equation}
				\rho_{M}\equiv\int_{u\in M'}du\,u\rho u^{\dagger}
			\end{equation}
			where the integration is over the set of unitary elements $u\in M'$, using the invariant Haar measure $du$ on this compact group. To see the unitary subgroup is indeed compact, recall that the unitary group is compact in general so any Cauchy convergent sequence of unitaries $u_{n}\in M'$ will converge to some unitary $u$, and by continuity of the commutator the limit $u$ will also be in $M'$.\\
			Defined this way, $\rho_{M}$ is clearly:
			\begin{itemize}
				\item Hermitian, $\rho_{M}^{\dagger}=\rho_{M}$.
				\item Non-negative.
				\item Has trace one, $\text{Tr}(\rho_{M})=\text{Tr}(u\rho u^{\dagger})=1$.
			\end{itemize}
			so is indeed a state. To show $\rho_{M}\in M$, we show that it commutes with any unitary $v\in M'$, and thus is in $M''=M$ by the bicommutant theorem and the fact that von Neumann algebras are spanned by their unitary elements. So fix some unitary $v\in M'$. Then we have
			\begin{equation}
				v\rho_{M}=\int_{u\in M'}du\,vu\rho u^{\dagger}=\int_{u'\in M'}du'\,u'\rho u'^{\dagger}v=\rho_{M}v
			\end{equation}
			where in the second equality we set $u'=vu\implies du'=vdu$, and use invariance of the measure.\\
			The last thing to do is to show $\rho_{M}$ is unique. Suppose there existed $\rho_{M}'\neq\rho_{M}$ obeying all the results of the theorem. Then, we must have $\text{Tr}((\rho_{M}-\rho_{M}')x)=0$ for all $x\in M$. But if we take $x=\rho_{M}-\rho_{M}'$, we get $\text{Tr}(\rho_{M}-\rho_{M}')^{2}=0$, which implies $\rho_{M}=\rho_{M}'$.
		\end{proof}
		\item This theorem essentially tells us that to compute expectations in $M$, we can always replace any state by an element of $M$
		\item Let's consider the case where $M$ is a factor as an example - what is $\rho_{M}$?
		\item By theorem 1.5, we know there exists a factorisation $\mathcal{H}=\mathcal{H}_{A}\otimes \mathcal{H}_{\overline{A}}$ such that $M=\mathcal{L}(\mathcal{H}_{A})\otimes I_{\overline{A}}$
		\item If we define the reduced state
		\begin{equation}
			\rho_{A}\equiv \text{Tr}_{\overline{A}}(\rho)
		\end{equation}
		then the operator
		\begin{equation}
			\rho_{M}\equiv \rho_{A}\otimes\frac{I_{\overline{A}}}{|\overline{A}|}
		\end{equation}
		obeys the results of theorem 1.7
		\item By uniqueness, the $\rho_{M}$ defined here must be equivalent to (1.17)
		\item We can do the same procedure for a more general $M$ which isn't necessarily a factor
		\item We know that there's a decomposition 
		\begin{equation}
			\mathcal{H}=\oplus_{\alpha}\left(\mathcal{H}_{A_{\alpha}}\otimes\mathcal{H}_{\overline{A}_{\alpha}}\right)
		\end{equation}
		in terms of which 
		\begin{equation}
			M=\oplus_{\alpha}\left(\mathcal{L}\left(\mathcal{H}_{A_{\alpha}}\right)\otimes I_{\overline{A}_{\alpha}}\right)
		\end{equation}
		\item Any state $\rho$ can be written in block form wrt (1.21), and only blocks on the $\alpha$ diagonal contribute to expectations of $M$
		\item On each diagonal block, we can define
		\begin{equation}
			p_{\alpha}\rho_{A_{\alpha}}\equiv \text{Tr}_{\overline{A}}(\rho_{\alpha\alpha})
		\end{equation}
		where $p_{\alpha}$ is a positive number chosen so $\text{Tr}_{A_{\alpha}}(\rho_{A_{\alpha}})=1$
		\item The condition $\text{Tr}(\rho)=1$ implies that $\sum_{\alpha}p_{\alpha}=1$
		\item We can then finally define the block diagonal state
		\begin{equation}
			\rho_{M}\equiv\oplus_{\alpha}\left(p_{\alpha}\rho_{A_{\alpha}}\otimes\frac{I_{\overline{A}_{\alpha}}}{|\overline{A}_{\alpha}|}\right)
		\end{equation}
		which again obeys theorem 1.7
	\end{itemize}
	\subsubsection{Example}
	\begin{itemize}
		\item As an example for a simple case of a factor algebra, consider $\mathcal{H}=\mathcal{H}_{2}\otimes\mathcal{H}_{2}$, so $M=\mathcal{L}(\mathcal{H}_{2})\otimes I_{2}$ is a factor
		\item In other words, an arbitrary $x\in M$ looks like
		\begin{equation}
			x=\begin{pmatrix}x_{11}&0&x_{12}&0\\0&x_{11}&0&x_{12}\\x_{21}&0&x_{22}&0\\0&x_{21}&0&x_{22}\end{pmatrix}
		\end{equation}
		\item Consider the state defined by
		\begin{equation}
			\rho=\begin{pmatrix}
				1/2&1/4&0&1/4\\1/4&1/4&0&0\\0&0&0&0\\1/4&0&0&1/4
			\end{pmatrix}
		\end{equation}
		which gives $\mathbb{E}_{\rho}(x)=\frac{3}{4}x_{11}+\frac{1}{4}x_{22}$
		\item Following the procedure, we calculate the reduced state:
		\begin{equation}
			\rho_{A}=\begin{pmatrix}
				3/4&0\\0&1/4
			\end{pmatrix}
		\end{equation}
		and so 
		\begin{equation}
			\rho_{M}=\begin{pmatrix}
				3/8&0&0&0\\0&3/8&0&0\\0&0&1/8&0\\0&0&0&1/8
			\end{pmatrix}
		\end{equation}
		from which it is easy to find that $\mathbb{E}_{\rho_{M}}(x)=\frac{3}{4}x_{11}+\frac{1}{4}x_{22}$ as expected
	\end{itemize}
	\subsection{Modified Trace and Entropy}
	\begin{itemize}
		\item From (1.20), we see that $\rho_{M}$ is closely related to $\rho_{A}$, and $\rho_{A}$ is what is used to define the von Neumann entropy of $\rho$ on system $A$/factor $M$ via $S(\rho_{A})=-\text{Tr}_{A}(\rho_{A}\log{\rho_{A}})$
		\item This gives a natural generalisation for the entropy of a state $\rho$ on an arbitrary von Neumann algebra $M$:
		\begin{equation}
			S(\rho,M)\equiv-\sum_{\alpha}\text{Tr}_{A_{\alpha}}\left(p_{\alpha}\rho_{A_{\alpha}}\log(p_{\alpha}\rho_{A_{\alpha}})\right)=-\sum_{\alpha}p_{\alpha}\log{p_{\alpha}}+\sum_{\alpha}p_{\alpha}S(\rho_{A_{\alpha}})
		\end{equation}
		\item Let's arrive at this entropy from a more abstract perspective
		\item It would be ideal to extract this entropy directly from the state $\rho_{M}$, but there's some issues
		\item For example, suppose we just set the entropy of $\rho$ on $M$ to be $S(\rho_{M})$; even when $M$ is a factor, then
		\begin{equation}
			S(\rho_{M})=-\text{Tr}(\rho_{M}\log{\rho_{M}})=-\text{Tr}\left(\rho_{A}\otimes\frac{I_{\overline{A}}}{|\overline{A}|}\right)\log\left(\rho_{A}\otimes\frac{I_{\overline{A}}}{|\overline{A}|}\right)=S(\rho_{A})+\log|\overline{A}|
		\end{equation}
		which disagrees with what we'd want by $\log|\overline{A}|$
		\item In the general case, the disagreement is $\sum_{\alpha}p_{\alpha}\log|\overline{A}_{\alpha}|$
		\item The issue stems from $\rho_{M}$ being supported on the full Hilbert space; we haven't taken the partial trace, so the entropy from the $|\overline{A}_{\alpha}|$ system is contributing
		\item One way to deal with this is to introduce a \textit{modified trace}
		\item For a von Neumman algebra $M$ on $\mathcal{H}$, the trace of a minimal projection is usually not one itself
		\item For example, if $M$ is a factor, then any minimal projection has the form $\ketbras{v}_{A}\otimes I_{\overline{A}}$, which has trace $|\overline{A}|$
		\item The entropy of this state on $\mathcal{H}_{A}$ is clearly 0, which can be seen by just tracing out $\overline{A}$ - but can we see this from a perspective more inherent to $M$ rather than the underlying systems?
		\item One canonical way to do this is to define a \textit{normalised trace} on $M$, so $\hat{\text{Tr}}p=1$ for any minimal projection $p\in M$; this means 
		\begin{equation}
			\ntr\equiv\frac{1}{|\overline{A}|}\text{Tr}
		\end{equation}
		\item If we further define
		\begin{equation}
			\hat{\rho}_{M}\equiv |\overline{A}|\rho_{M}
		\end{equation}
		then for any $x\in M$, we have
		\begin{equation}
			\mathbb{E}_{\rho}(x)=\text{Tr}\rho x=\text{Tr}\rho_{M}x=\ntr\hat{\rho}_{M}x
		\end{equation}
		\item From (1.20), we can then define the entropy $S(\rho,M)$ by using the normalised trace and normalised $\hat{\rho}_{M}$:
		\begin{equation}
			S(\rho,M)\equiv-\ntr\hat{\rho}_{M}\log\hat{\rho}_{M}=-\text{Tr}_{A}\rho_{A}\log\rho_{A}=S(\rho_{A})
		\end{equation}
		which gives an `intrinsic' definition of the entropy of $\rho$ on a factor $M$
		\item It is just the expectation value of the operator $-\log\hat{\rho}_{M}$ in the state $\rho_{M}$
		\item In the case where $M$ is not a factor, it's a bit trickier
		\item We just define a normalised trace $\ntr$ again so that $\ntr{p}=1$ for any minimal projection $p\in M$; however, this time it turns out that $\ntr$ is \textbf{not} proportional to $\text{Tr}$
		\item We are only interested in defining $\ntr$ for elements of $M$; recalling the block decomposition (1.21), no elements of $M$ mix between different blocks, so we can normalise the trace independently in each block without breaking the usual defining characteristic property of trace that $\ntr{xy}=\ntr{yx},\;\forall x,y\in M$
		\item We can then just define $\ntr$ as the unique linear operation on $M$ such that $\ntr{xy}=\ntr{yx},\;\forall x,y\in M$, which also gives $\ntr{p}=1$ for any minimal projection $p$
		\item Explicitly in terms of decompositions (1.21) and (1.22), if $x\in M$ has representation
		\begin{equation}
			x=\oplus_{\alpha}\left(x_{\alpha}\otimes I_{\overline{A}_{\alpha}}\right)
		\end{equation}
		then
		\begin{equation}
			\ntr{x}=\sum_{\alpha}\ntr_{\alpha}\left(x_{\alpha}\otimes I_{\overline{A}_{\alpha}}\right)=\sum_{\alpha}\text{Tr}_{A_{\alpha}}x_{\alpha}
		\end{equation}
		\item Moreover, given any $\rho_{M}\in M$, we can introduce $\hat{\rho}_{M}$ again defined so that for any $x\in M$, we have
		\begin{equation}
			\mathbb{E}_{\rho}(x)=\text{Tr}\rho x=\text{Tr}\rho_{M}x=\ntr\hat{\rho}_{M}x
		\end{equation}
		\item Explicitly, from (1.24) we have
		\begin{equation}
			\hat{\rho}_{M}=\oplus_{\alpha}\left(p_{\alpha}\rho_{A_{\alpha}}\otimes I_{\overline{A}_{\alpha}}\right)
		\end{equation}
		\item At long last, we can define the entropy of $\rho$ on algebra $M$ by
		\begin{equation}
			S(\rho,M)\equiv-\ntr\hat{\rho}_{M}\log\hat{\rho}_{M}
		\end{equation}
		which can be shown to be equivalent to (1.29)
	\end{itemize}
	\subsection{Properties of Entropy}
	\begin{itemize}
		\item For ease of reference, the entropy of $\rho$ on $M$ is
		\begin{equation}
			S(\rho,M)\equiv-\sum_{\alpha}\text{Tr}_{A_{\alpha}}\left(p_{\alpha}\rho_{A_{\alpha}}\log(p_{\alpha}\rho_{A_{\alpha}})\right)=-\sum_{\alpha}p_{\alpha}\log{p_{\alpha}}+\sum_{\alpha}p_{\alpha}S(\rho_{A_{\alpha}})
		\end{equation}
		\item Schematically, this has a `classical' piece given by the Shannon entropy of the probability distribution $p_{\alpha}$ for the center $Z_{M}$, and a `quantum' piece given by the weighted average of the von Neumann entropy of each block over this distribution
		\item It has the following properties:
		\begin{itemize}
			\item $S(\rho,M)=S(u\rho u^{\dagger},M)$ for any unitary $u\in M$
			\item $S(\rho,M)\geq 0$, with equality iff $\rho_{M}$ is a minimal projection
			\item $S(\rho,M)\leq\log\left(\ntr I\right)=\log\left(\sum_{\alpha}|A_{\alpha}|\right)$, with equality iff $\rho_{A_{\alpha}}=I_{A_{\alpha}}/|A_{\alpha}|$ and $p_{\alpha}=|A_{\alpha}|/\sum_{\beta}|A_{\beta}|$
			\item $S\left(\sum_{i}\lambda_{i}\rho_{i}\right)\geq\sum_{i}\lambda_{i}S(\rho_{i})$, where the $\rho_{i}$ are any set of states and $\sum_{i}\lambda_{i}=1$
			\item If $\rho$ is pure, then $S(\rho,M)=S(\rho,M')$
		\end{itemize}
		\item We also have an analogous definition for relative entropy
		\begin{definition}[Relative Entropy]
			Given two states $\rho,\sigma\in M$, the \textbf{relative entropy} between them is
			\begin{equation}
				\begin{aligned}
					S(\rho|\sigma,M)&\equiv\ntr\left(\hat{\rho}_{M}\log\hat{\rho}_{M}-\hat{\rho}_{M}\log\hat{\sigma}_{M}\right)\\&=-S(\rho,M)+\mathbb{E}_{\rho}(-\log\hat{\sigma}_{M})\\&=\sum_{\alpha}p_{\alpha}^{\{\rho\}}\log\frac{p_{\alpha}^{\{\rho\}}}{p_{\alpha}^{\{\sigma\}}}+\sum_{\alpha}p_{\alpha}^{\{\rho\}}S(\rho_{A_{\alpha}}|\sigma_{A_{\alpha}})
				\end{aligned}
			\end{equation}
			\item This again has a classical contribution which measures the distinguishability of distributions $p_{\alpha}^{\{\rho\}}$ and $p_{\alpha}^{\{\sigma\}}$ on the center $Z_{M}$, and a quantum piece averaging the relative entropy on each block
			\item As usual, $S(\rho|\sigma,M)\geq 0$ with equality iff $\rho_{M}=\sigma_{M}$
		\end{definition}
	\end{itemize}
\end{document}