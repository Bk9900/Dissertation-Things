\documentclass[12pt,a4paper]{article}
\usepackage[utf8]{inputenc}
\usepackage[T1]{fontenc}
\usepackage{amsmath}
\usepackage{amsfonts}
\usepackage{amssymb}
\usepackage{amsthm}
\usepackage{graphicx}
\usepackage{geometry}
\usepackage{mathrsfs}
\usepackage{braket}
\usepackage{simpler-wick}
\usepackage{simplewick}
\usepackage{tikz}
\usetikzlibrary{quantikz}
\usepackage{tikz-feynman}
\usepackage{subcaption}
\usepackage{bm}
\usepackage{slashed}
\usepackage{tensor}
\usepackage{hyperref}
\usepackage{mathtools}
\usepackage{float}
\usepackage{nicematrix}
%\usepackage{pst-node}
%\usepackage{auto-pst-pdf}
\title{Week 4 - Harlow 3 in Pollack Language}
\author{Ben Karsberg}
\date{2021-22}
\newgeometry{vmargin={15mm}, hmargin={20mm,20mm}}
\numberwithin{equation}{section}
\newcommand{\ketbra}[2]{\ket{#1}\bra{#2}}
\newcommand{\ketbras}[1]{\ketbra{#1}{#1}}
\newcommand{\Pc}{P_{\text{code}}}
\newcommand{\Hcode}{\mathcal{H}_{\text{code}}}
\newcommand{\ntr}{\hat{\text{Tr}}}
\newcommand{\gen}[1]{\braket{#1}_{vN}}
\newcommand{\ol}[1]{\overline{#1}}
\newcommand{\mdef}{V^{\dagger}(\mathcal{L}(\mathcal{H}_{A})\otimes I_{\overline{A}})V}
\theoremstyle{definition}
\newtheorem{definition}{Definition}[section]
\theoremstyle{theorem}
\newtheorem{theorem}{Theorem}[section]
\theoremstyle{example}
\newtheorem{example}{Example}[section]
\newtheorem{proposition}{Proposition}
\begin{document}
	\maketitle
	\section{von Neumann Algebras - Relevant Bits}
	\begin{itemize}
		\item The von Neumann classification theorem tells us that for any von Neumann algebra $M$ on $\Hcode$, we have a decomposition
		\begin{equation}
			\Hcode=\oplus_{\alpha}(\mathcal{H}_{a_{\alpha}}\otimes\mathcal{H}_{\overline{a}_{\alpha}})
		\end{equation}
		where $M$ is given by the set of all operators $\tilde{O}$ which are block diagonal in $\alpha$, and that within each block they act as $\tilde{O}_{a_{\alpha}}\otimes I_{\overline{a}_{\alpha}}$
		\item In matrix form:
		\begin{equation}
			\tilde{O}=\begin{pmatrix}
				\tilde{O}_{a_{1}}\otimes I_{\overline{a}_{1}}&0&\dots\\0&\tilde{O}_{a_{2}}\otimes I_{\ol{a}_{2}}&\dots\\\vdots&\vdots&\ddots
			\end{pmatrix}
		\end{equation}
		for any $\tilde{O}\in M$
		\item The \textit{commutant} of $M$, denoted $M'$, is the set of all operators which commute with $M$, which are block-diagonal of the form
		\begin{equation}
			\tilde{O}'=\begin{pmatrix}
				I_{a_{1}}\otimes\tilde{O}'_{\ol{a}_{1}}&0&\dots\\0&I_{a_{2}}\otimes\tilde{O}'_{\ol{a}_{2}}\dots\\\vdots&\vdots&\ddots
			\end{pmatrix}
		\end{equation}
		\item The \textit{center} $Z_{M}\equiv M\cap M'$ consists of operators of the form
		\begin{equation}
			\tilde{\Lambda}=\begin{pmatrix}\lambda_{1}(I_{a_{1}}\otimes I_{\ol{a}_{1}})&0&\dots\\0&\lambda_{2}(I_{a_{2}}\otimes I_{\ol{a}_{2}})&\dots\\\vdots&\vdots&\ddots\end{pmatrix}
		\end{equation}
		where the $\lambda_{\alpha}\in\mathbb{C}$
		\item When $M$ is the set of all operators on a tensor factor, $Z_{M}$ is trivial and $M$ is called a \textit{factor}
		\item Harlow introduces orthonormal bases $\ket{\widetilde{\alpha,i}}$ and $\ket{\widetilde{\alpha,\ol{i}}}$ for $\mathcal{H}_{a_{\alpha}}$ and $\mathcal{H}_{\ol{a}_{\alpha}}$ respectively
		\item This gives us an orthonormal basis for the full $\Hcode$ space:
		\begin{equation}
			\ket{\widetilde{\alpha,i\ol{i}}}\equiv\ket{\widetilde{\alpha,i}}\otimes\ket{\widetilde{\alpha,\ol{i}}}
		\end{equation}
		\item Given state $\tilde{\rho}$ and von Neumann algebra $M$ on $\Hcode$, we can define an algebraic entropy for $\tilde{\rho}$ on $M$ which reduces to standard von Neumann entropy when $M$ is a factor
		\item We first define
		\begin{equation}
			p_{\alpha}\tilde{\rho}_{a_{\alpha}}\equiv\text{Tr}_{\ol{a}_{\alpha}}\tilde{\rho}_{\alpha\alpha}
		\end{equation}
		where $\tilde{\rho}_{\alpha\alpha}$ are the diagonal blocks of $\tilde{\rho}$, and $p_{\alpha}\in[0,1]$ chosen so $\text{Tr}_{a_{\alpha}}\tilde{\rho}_{a_{\alpha}}=1$
		\item This further implies $\sum_{\alpha} p_{\alpha}=1$
		\item We then define algebraic entropy as
		\begin{equation}
			S(\tilde{\rho},M)\equiv -\sum_{\alpha}p_{\alpha}\log{p_{\alpha}}+\sum_{\alpha}p_{\alpha}S(\tilde{\rho}_{\alpha})
		\end{equation}
		\item We similarly define the entropy of $\tilde{\rho}$ on $M'$ via
		\begin{equation}
			p_{\alpha}\tilde{\rho}_{\ol{a}_{\alpha}}\equiv\text{Tr}_{a_{\alpha}}\tilde{\rho}_{\alpha\alpha}
		\end{equation}
		and then
		\begin{equation}
			S(\tilde{\rho},M')\equiv-\sum_{\alpha}p_{\alpha}\log{p_{\alpha}}+\sum_{\alpha}p_{\alpha}S(\tilde{\rho}_{\ol{a}_{\alpha}})
		\end{equation}
	\end{itemize}
	\subsection{Harlow Theorem 3}
	\begin{itemize}
		\item This is Harlow's most general theorem on error correction
		\begin{theorem}
			Suppose $\mathcal{H}$ is a finite dimensional Hilbert space, which has a tensor product structure $\mathcal{H}=\mathcal{H}_{A}\otimes\mathcal{H}_{\ol{A}}$, and suppose $\Hcode\subseteq\mathcal{H}$ is a subspace on which we have a von Neumann algebra $M$. Set $\ket{\widetilde{\alpha,i\ol{i}}}$ to be an orthonormal basis of $\Hcode$ which is compatible with the block decomposition induced by $M$. Also define $\ket{\phi}\equiv\frac{1}{\sqrt{|R|}}\sum_{\alpha,i,\ol{i}}\ket{\alpha,i\ol{i}}_{R}\ket{\widetilde{\alpha, i\ol{i}}}_{A\ol{A}}$, where $R$ is an auxiliary system with $\mathcal{H}_{R}=\Hcode$. Then, the following 4 statements are equivalent:
			\begin{enumerate}
				\item For any operator $\tilde{O}\in M$, there exists an operator $O_{A}$ on $\mathcal{H}_{A}$ with an equivalent action on $\Hcode$; that is, for any $\ket{\tilde{\psi}}\in\Hcode$, we have
				\begin{equation}
					\begin{aligned}
						O_{A}\ket{\tilde{\psi}}&=\tilde{O}\ket{\tilde{\psi}}\\
						O_{A}^{\dagger}\ket{\tilde{\psi}}&=\tilde{O}^{\dagger}\ket{\tilde{\psi}}
					\end{aligned}
				\end{equation}
				\item For any operator $X_{\ol{A}}$ on $\mathcal{H}_{\ol{A}}$, we have
				\begin{equation}
					\Pc X_{\ol{A}}\Pc=X'\Pc
				\end{equation}
				where $X'\in M'$, and $\Pc$ is the projector onto $\Hcode$.
				\item For any $\tilde{O}\in M$, we have
				\begin{equation}
					[O_{R},\rho_{R\ol{A}}(\phi)]=0
				\end{equation}
				where $O_{R}$ is the unique operator on $\mathcal{H}_{R}$ such that
				\begin{equation}
					\begin{aligned}
						O_{R}\ket{\phi}&=\tilde{O}\ket{\phi}\\
						O_{R}^{\dagger}\ket{\phi}&=\tilde{O}^{\dagger}\ket{\phi}
					\end{aligned}
				\end{equation}
				which acts with the same matrix elements on $R$ as $\tilde{O}^{T}$ does on $\Hcode$.
				\item $\sum_{\alpha}|\mathcal{H}_{a_{\alpha}}|\leq|\mathcal{H}_{A}|$, and we can decompose 
				\begin{equation} 
					\mathcal{H}_{A}=\oplus_{\alpha}(\mathcal{H}_{A_{1}^{\alpha}}\otimes\mathcal{H}_{A_{2}^{\alpha}})\oplus\mathcal{H}_{A_{3}}
				\end{equation}
				with $|A_{1}^{\alpha}|=|a_{\alpha}|$, and there exists a unitary $U_{A}$ on $\mathcal{H}_{A}$ and sets of orthonormal states $\ket{\chi_{\alpha,\ol{i}}}_{A_{2}^{\alpha}\ol{A}}\in\mathcal{H}_{A_{2}^{\alpha}\ol{A}}$ such that
				\begin{equation}
					\ket{\widetilde{\alpha,i\ol{i}}}=U_{A}\bigg(\ket{\alpha,i}_{A_{1}^{\alpha}}\otimes\ket{\chi_{\alpha,\ol{i}}}_{A_{2}^{\alpha}\ol{A}}\bigg)
				\end{equation}
				where $\ket{\alpha,i}_{A_{1}^{\alpha}}$ is an orthonormal basis for $\mathcal{H}_{A_{1}^{\alpha}}$.
			\end{enumerate}
		\end{theorem}
		\item This theorem characterises how well a code subspace can correct a subalgebra $M$ for the erasure of $\ol{A}$
		\item It reduces to subsystem correction if $M$ is a factor, and to standard correction of $M=\mathcal{L}(\Hcode)$
		\begin{proof}
			$(1)\implies(2)$: Suppose for sake of contradiction that $\Pc X_{\ol{A}}\Pc=x'\Pc$ where $x'$ is an operator on $\Hcode$ but is not in $M'$. Then there must exist an $\tilde{O}\in M$ and a state $\ket{\tilde{\psi}}\in\Hcode$ such that
			\begin{equation}
				\braket{\tilde{\psi}|[x',\tilde{O}]|\tilde{\psi}}=\braket{\tilde{\psi}|[X_{\ol{A}},\tilde{O}]|\tilde{\psi}}\neq 0
			\end{equation}
			but such an $\tilde{O}$ cannot have a corresponding $O_{A}$ as such an $O_{A}$ would automatically commute with $X_{\ol{A}}$.\\
			$(2)\implies (3)$: Say $\tilde{O}\in M$, and $X_{\ol{A}}$ and $Y_{R}$ are arbitrary operators on $\mathcal{H}_{\ol{A}}$ and $\mathcal{H}_{R}$ respectively. We then have:
			\begin{equation}
				\begin{aligned}
					\text{Tr}_{R\ol{A}}\left(O_{R}\rho_{R\ol{A}}(\phi)X_{\ol{A}}Y_{R}\right)&=\braket{\phi|X_{\ol{A}}Y_{R}O_{R}|\phi}\\&=\braket{\phi|X_{\ol{A}}Y_{R}\tilde{O}|\phi}\\&=\braket{\phi|\tilde{O}X_{\ol{A}}Y_{R}|\phi}\\&=\braket{\phi|O_{R}X_{\ol{A}}Y_{R}|\phi}\\&=\text{Tr}_{R\ol{A}}(\rho_{R\ol{A}}(\phi)O_{R}X_{\ol{A}}Y_{R})
				\end{aligned}
			\end{equation}
			where the first equality is due to a small calculation, the second is by definition of $O_R$, the third is by the fact that $X_{\ol{A}}$ commutes with $\tilde{O}$ by (3), and the last is just a calculation. This can only hold for arbitrary $X_{\ol{A}}$ and $Y_{R}$ if $[O_{R},\rho_{R\ol{A}}(\phi)]=0$.\\
			$(3)\implies (4)$: Our basis $\ket{\alpha,i\ol{i}}_{R}$ for $\mathcal{H}_{R}$ gives a decomposition for $\mathcal{H}_{R\ol{A}}\equiv\mathcal{H}_{R}\otimes\mathcal{H}_{\ol{A}}$ as
			\begin{equation}
				\mathcal{H}_{R\ol{A}}=\oplus_{\alpha}\left(\mathcal{H}_{R_{\alpha}}\otimes\mathcal{H}_{\ol{R}_{\alpha}}\otimes\mathcal{H}_{\ol{A}}\right)
			\end{equation}
			From (3), we know that $[O_{R},\rho_{R\ol{A}}(\phi)]=0$ for all such $O_{R}$ as defined. This means that the reduced state of $\rho_{R\ol{A}}(\phi)$ on $R$ alone must be $I_{R}/|R|$. We therefore have
			\begin{equation}
				\rho_{R\ol{A}}(\phi)=\oplus_{\alpha}\left[\frac{|R_{\alpha}||\ol{R}_{\alpha}|}{|R|}\left(\frac{I_{R_{\alpha}}}{|R_{\alpha}|}\otimes\rho_{\ol{R}_{\alpha}\ol{A}}\right)\right]
			\end{equation}
			for some state $\rho_{\ol{R}_{\alpha}\ol{A}}$, with the coefficient out the front to ensure this is a valid state satisfying $\text{Tr}_{R\ol{A}}(\rho_{R\ol{A}}(\phi))=1$. Moreover, since $\rho_{R}=I_{R}/|R|$, we must have $\text{Tr}_{\ol{A}}(\rho_{\ol{R}_{\alpha}\ol{A}})=I_{\ol{R}_{\alpha}}/|\ol{R}_{\alpha}|$.\\
			Now, since $\ket{\phi}$ purifies $\rho_{R\ol{A}}$ on $A$, and in a purification the dimension of the purifying system is necessarily as big as the rank of the state being purified (cf Schmidt decomposition), so denoting the rank of $\rho_{\ol{R}_{\alpha}\ol{A}}$ as $|\rho_{\ol{R}_{\alpha}\ol{A}}|$, we have
			\begin{equation}
				\sum_{\alpha}|R_{\alpha}||\rho_{\ol{R}_{\alpha}\ol{A}}|\leq |A|
			\end{equation}
			This means we can indeed decompose
			\begin{equation}
				\mathcal{H}_{A}=\oplus_{\alpha}\left(\mathcal{H}_{A_{1}^{\alpha}}\otimes\mathcal{H}_{A_{2}^{\alpha}}\right)\oplus\mathcal{H}_{A_{3}}
			\end{equation}
			where $|A_{1}^{\alpha}|=|R_{\alpha}|=|a_{\alpha}|$ and $|A_{2}^{\alpha}|\geq|\rho_{\ol{R}_{\alpha}\ol{A}}|$. For each $\alpha$, we can therefore purify $\rho_{\ol{R}_{\alpha}\ol{A}}$ on $A_{2}^{\alpha}$, and since $\text{Tr}_{\ol{A}}(\rho_{\ol{R}_{\alpha}\ol{A}})=I_{\ol{R}_{\alpha}}/|\ol{R}_{\alpha}|$, this purification looks like
			\begin{equation}
				\ket{\psi_{\alpha}}_{\ol{R}_{\alpha}A_{2}^{\alpha}\ol{A}}=\frac{1}{\sqrt{|\ol{R}_{\alpha}|}}\sum_{\ol{i}}\ket{\alpha,\ol{i}}_{\ol{R}_{\alpha}}\ket{\chi_{\alpha,\ol{i}}}_{A_{2}^{\alpha}\ol{A}}
			\end{equation}
			where the $\ket{\chi_{\alpha,\ol{i}}}$s are mutually orthonormal on $A_{2}^{\alpha}\ol{A}$. This means a purification for $\rho_{R\ol{A}}$ on the full $A$ system is
			\begin{equation}
				\ket{\phi'}=\sum_{\alpha,i\ol{i}}\frac{1}{\sqrt{|R_{\alpha}|}}\ket{\alpha,i}_{R_{\alpha}}\ket{\alpha,i}_{A_{1}^{\alpha}}\ket{\psi_{\alpha}}_{\ol{R}_{\alpha}A_{2}^{\alpha}\ol{A}}=\frac{1}{\sqrt{|R|}}\sum_{\alpha,i\ol{i}}\ket{\alpha,i\ol{i}}_{R}\ket{\alpha,i}_{A_{1}^{\alpha}}\ket{\chi_{\alpha,\ol{i}}}_{A_{2}^{\alpha}\ol{A}}
			\end{equation}
			and since $\ket{\phi}$ and $\ket{\phi'}$ are two purifications of $\rho_{R\ol{A}}$ on $A$, they must differ by some unitary $U_{A}$.\\
			$(4)\implies (1)$: Just define
			\begin{equation}
				O_{A}\equiv U_{A}(\oplus_{\alpha}(O_{A_{1}^{\alpha}}\otimes I_{A_{2}^{\alpha}}))U_{A}^{\dagger}
			\end{equation}
			where $O_{A_{1}^{\alpha}}$ acts on $\mathcal{H}_{A_{1}^{\alpha}}$ in the same way as $\tilde{O}_{a_{\alpha}}$ from (1.2) does on $\mathcal{H}_{a_{\alpha}}$.
		\end{proof}
		\item Harlow defines a \textit{subalgebra code with complementary recovery} as one where not only can we represent any element of $M$ on $A$ as in $(1)$, but we can also represent any element of $M'$ on $\ol{A}$
		\item Equivalence of (1) and (4) then tells us that
		\begin{equation}
			\ket{\widetilde{\alpha,i\ol{i}}}=U_{A}U_{\ol{A}}\bigg(\ket{\alpha,i}_{A_{1}^{\alpha}}\ket{\alpha,\ol{i}}_{\ol{A}_{1}^{\alpha}}\ket{\chi_{\alpha}}_{A_{2}^{\alpha}\ol{A}_{2}^{\alpha}}\bigg)
		\end{equation}
		where we've decomposed
		\begin{equation}
			\mathcal{H}_{\ol{A}}=\oplus_{\alpha}\bigg(\mathcal{H}_{\ol{A}_{1}^{\alpha}}\otimes\mathcal{H}_{\ol{A}_{2}^{\alpha}}\bigg)\oplus\mathcal{H}_{\ol{A}_{3}}
		\end{equation}
		where $|\ol{A}_{1}^{\alpha}|=|\ol{a}_{\alpha}|$
	\end{itemize}
	\subsection{RT Formula Stuff}
	\begin{itemize}
		\item Consider arbitrary encoded state $\tilde{\rho}$ in a subalgebra code with complementary recovery on $A$ and $\ol{A}$
		\item Recall that in order to define an algebraic entropy of $\tilde{\rho}$ on $M$, we considered the diagonal blocks $\tilde{\rho}_{\alpha\alpha}$ of $\tilde{\rho}$ and defined
		\begin{equation}
			p_{\alpha}\tilde{\rho}_{a_{\alpha}}\equiv\text{Tr}_{\ol{a}_{\alpha}}\tilde{\rho}_{\alpha\alpha}
		\end{equation}
		so that
		\begin{equation}
			S(\tilde{\rho},M)\equiv-\sum_{\alpha}p_{\alpha}\log{p_{\alpha}}+\sum_{\alpha}p_{\alpha}S(\tilde{\rho}_{a_{\alpha}})
		\end{equation}
		\item Now, consider (1.25): we can rewrite it as
		\begin{equation}
			\ket{\widetilde{\alpha,i\ol{i}}}=\ket{\widetilde{\alpha,i}}_{a_{\alpha}}\otimes\ket{\widetilde{\alpha,\ol{i}}}_{\ol{a}_{\alpha}}=U_{A}U_{\ol{A}}\bigg(\ket{\alpha,i}_{A_{1}^{\alpha}}\ket{\alpha,\ol{i}}_{\ol{A}_{1}^{\alpha}}\ket{\chi_{\alpha}}_{A_{2}^{\alpha}\ol{A}_{2}^{\alpha}}\bigg)
		\end{equation}
		\item So, if we define $\rho_{A_{1}^{\alpha}}$ and $\rho_{\ol{A}_{1}^{\alpha}}$ to have the same matrix elements on $\mathcal{H}_{A_{1}^{\alpha}}$ and $\mathcal{H}_{\ol{A}_{1}^{\alpha}}$ as $\tilde{\rho}_{a_{\alpha}}$ and $\tilde{\rho}_{\ol{a}_{\alpha}}$ do on $\mathcal{H}_{a_{\alpha}}$ and $\mathcal{H}_{\ol{a}_{\alpha}}$, we see that for arbitrary $\tilde{\rho}$ with diagonal blocks $\tilde{\rho}_{\alpha\alpha}$, we have
		\begin{equation}
			\tilde{\rho}_{\alpha\alpha}=p_{\alpha}\tilde{\rho}_{a_{\alpha}}\otimes\tilde{\rho}_{\ol{a}_{\alpha}}=U_{A}U_{\ol{A}}\bigg(p_{\alpha}\rho_{A_{1}^{\alpha}}\otimes\rho_{A_{2}^{\alpha}}\otimes \ketbras{\chi_{\alpha}}\bigg)U_{\ol{A}}^{\dagger}U_{A}^{\dagger}
		\end{equation}
		\item We therefore obtain reduced density matrices on $A$ and $\ol{A}$ as
		\begin{equation}
			\begin{aligned}
				\tilde{\rho}_{A}&\equiv\text{Tr}_{\ol{A}}\tilde{\rho}=U_{A}\bigg(\oplus_{\alpha}(p_{\alpha}\rho_{A_{1}^{\alpha}}\otimes\chi_{A_{2}^{\alpha}})\bigg)U_{A}^{\dagger}\\
				\tilde{\rho}_{\ol{A}}&\equiv\text{Tr}_{A}\tilde{\rho}=U_{\ol{A}}\bigg(\oplus_{\alpha}(p_{\alpha}\rho_{\ol{A}_{1}^{\alpha}}\otimes\chi_{\ol{A}_{2}^{\alpha}})\bigg)U_{\ol{A}}^{\dagger}
			\end{aligned}
		\end{equation}
		where $\chi_{A_{2}^{\alpha}}=\text{Tr}_{\ol{A}_{2}^{\alpha}}\ketbras{\chi_{\alpha}}$ etc
		\item So, defining area operator
		\begin{equation}
			\mathcal{L}_{A}\equiv\oplus_{\alpha}S(\chi_{A_{2}^{\alpha}})I_{a_{\alpha}\ol{a}_{\alpha}}
		\end{equation}
		we find Ryu-Takayanagi formulae
		\begin{equation}
			\begin{aligned}
				S(\tilde{\rho}_{A})&=\text{Tr}\tilde{\rho}\mathcal{L}_{A}+S(\tilde{\rho},M)\\
				S(\tilde{\rho}_{\ol{A}})&=\text{Tr}\tilde{\rho}\mathcal{L}_{A}+S(\tilde{\rho},M')
			\end{aligned}
		\end{equation}
		\item Note that the area operator is now non-trivial, since $S(\chi_{A_{2}^{\alpha}})$ can take different values for different $\alpha$
		\item Moreover, $\mathcal{L}_{A}$ is clearly in $Z_{M}$
		\item Harlow also shows that obeying RT formulae in this way also implies complementary recovery, so in some sense complementary recovery and this `two-sided RT formula' for $A$ and $\ol{A}$ are equivalent
		\item Pollack shows that the existence of complementary recovery determines the von Neumann algebra $M$ uniquely - so why don't we do away with $M$?
		\item He suggests the possibility of a `one-sided RT formula' obeyed by $A$ but not $\ol{A}$ - is this mathematically possible?
		\item Might give this a go lol
	\end{itemize}
	\section{The Pollack Way}
	\subsection{Complementary Recovery}
	\begin{itemize}
		\item Harlow thinks of quantum erasure correction in terms of a subspace $\Hcode\subseteq\mathcal{H}$; Pollack et al. work instead in terms of a \textit{logical space} $\mathcal{H}_{L}$ which is thought of as being separate from $\mathcal{H}$ with the same dimensionality as $\Hcode$
		\item An \textit{encoding isometry} $V\,:\,\mathcal{H}_{L}\to\mathcal{H}$ then takes logical states and encodes them in the physical Hilbert space
		\item The image of $V$ is $\Hcode$
		\item We intuitively think of this as fixing a basis for $\Hcode$ as $\Hcode$ is invariant under the action $V\to VU_{L}$ for some unitary $U_{L}$ on $\mathcal{H}_{L}$
		\item This view makes the notion of the center of a von Neumann algebra on $\mathcal{H}_{L}$ simpler to understand, and in constructing explicit examples it is easier to write down $V$ rather than $\Hcode$
		\item In holography, there are two properties that are relevant for an RT formula: the subregion $A$ which determines entropy $S_{A}$ and the visible bulk degrees of freedom which determine the entropy $S_{\text{bulk},A}$
		\item The visible degrees of freedom are denoted by a von Neumann algebra $M$, and clearly $(V,A,M)$ are related, so we establish the following nomenclature
		\begin{definition}[Correctable and Private Algebra]
			Suppose $V\,:\,\mathcal{H}_{L}\to\mathcal{H}$ is an encoding isometry $V$ for some quantum error correcting code, and $A$ is a subregion of $\mathcal{H}$ inducing the factorisation $\mathcal{H}=\mathcal{H}_{A}\otimes\mathcal{H}_{\ol{A}}$. A von Neumann algebra $M\subseteq\mathcal{L}(\mathcal{H}_{L})$ is said to be:
			\begin{itemize}
				\item \textbf{Correctable} from $A$ with respect to $V$ if $M\subseteq V^{\dagger}(\mathcal{L}(\mathcal{H}_{A})\otimes I_{A})V$; for every $O_{L}\in M$ there exists an $O_{A}\in\mathcal{L}(\mathcal{H}_{A})$ such that $O_{L}=V^{\dagger}(O_{A}\otimes I_{\ol{A}})V$.
				\item \textbf{Private} from $A$ with respect to $V$ if $V^{\dagger}(\mathcal{L}(\mathcal{H}_{A})\otimes I_{\ol{A}})V\subseteq M'$; for every $O_{A}\in\mathcal{L}(\mathcal{H}_{A})$, $V^{\dagger}(O_{A}\otimes I_{\ol{A}})V$ commutes with every operator in $M$.
			\end{itemize}
		\end{definition}
		\item Intuitively: think of $M$ being correctable from $A$ if every operator in $M$ maps to a non-trivial operator on the tensor factor $A$ only, and private from $A$ if the preimage of any operator in $A$ commutes with all of $M$
		\item If $M$ is correctable, then it is a set of logical operators which can be performed on the encoded state given access to $A$ only
		\item A hermitian element of $M$ then corresponds to an observable on $\mathcal{H}_{L}$ which can be measured from $A$ - $M$ tells about which parts of a logical state can be recovered from knowledge of $A$ only
		\item Conversely, if $M$ is private then the observables in $M$ tell us which parts of a logical state are invisible from $A$
		\item We then can talk about \textit{complementary recovery} in this language
		\begin{definition}
			A code with encoding isometry $V\,:\,\mathcal{H}_{L}\to\mathcal{H}$, a subregion of the physical Hilbert space $A$, and a von Neumann algebra $M\subseteq\mathcal{L}(\mathcal{H}_{L})$, together taken as a triplet $(V,A,M)$ exhibit \textbf{complementary recovery} if:
			\begin{itemize}
				\item $M$ is correctable from $A$ with respect to $V$: $M\subseteq V^{\dagger}(\mathcal{L}(\mathcal{H}_{A})\otimes I_{\ol{A}})V$
				\item $M'$ is correctable from $\ol{A}$ with respect to $V$: $M'\subseteq V^{\dagger}(I_{A}\otimes \mathcal{L}(\mathcal{H}_{\ol{A}}))V$.
			\end{itemize}
		\end{definition}
		\item So far, this doesn't restrict the form $M$ can take very much: note that if $N\subset M$ is a subalgebra and $M$ is correctable, then so too is $N$
		\item It would therefore seem logical that if $(V,A,M)$ has complementary recovery, then so too does $(V,A,N)$
		\item We actually find that complementary recovery is so restrictive on $M$ that it gets determined uniquely, and such subalgebras $N$ do not exhibit complementary recovery
		\item This is important for holography because the von Neumann algebra $M$ is important in the RT formula: it tells us how to define the entropy of the bulk degrees of freedom visible from $A$ $S_{\text{bulk},A}$ via an algebraic entropy $S(M,A)$
		\item For this to make sense, $M$ needs to be uniquely determined by $V$ and the subregion $A$
		\item To show this, Pollack quotes the following result:
		\begin{theorem}[Correctability $\leftrightarrow$ Privacy]
			A von Neumann algebra $M$ is correctable from $A$ with respect to $V$ if and only if $M$ is private from $\ol{A}$ with respect to $V$.
		\end{theorem}
		\item The proof of this is quite straightforward
		\begin{proof}
			$(\implies)$: Suppose $\exists O_{\ol{A}}\in \mathcal{L}(\mathcal{H}_{\ol{A}})$ such that $V^{\dagger}(I_{A}\otimes I_{\ol{A}})V$ doesn't commute with some $O_{L}\in M$. But since $M$ is correctable from $A$, we know $\exists O_{A}\in\mathcal{L}(\mathcal{H}_{A})$ such that $O_{L}=V^{\dagger}(O_{A}\otimes I_{\ol{A}})V$, which clearly does commute with $V^{\dagger}(I_{A}\otimes I_{\ol{A}})V$, so such an $O_{\ol{A}}$ cannot exist and thus $M$ is private from $\ol{A}$ with respect to $V$.\\
			$(\impliedby)$: We know that for all $O_{\ol{A}}\in\mathcal{L}(\mathcal{H}_{\ol{A}})$, $V^{\dagger}(I_{A}\otimes O_{\ol{A}})V$ commutes with all $O_{L}\in M$. But this is only true for arbitrary $O_{L}$ and $O_{\ol{A}}$ if $O_{L}$ can be written as $O_{L}=V^{\dagger}(O_{A}\otimes I_{\ol{A}})V$ by Schur's lemma, which is just the definition of $M$ being correctable on $A$ with respect to $V$.
		\end{proof}
		\item This lemma essentially shows that for erasures, correctability is in some sense complementary to privacy: correctability on $A$ implies that $\ol{A}$ is erased
		\item Pollack says that informally, a subsystem $B$ of a Hilbert space $\mathcal{H}=A\otimes B$ is private if it completely decoheres after the action of a quantum channel
		\item Pollack now proves that the von Neumann algebra $M$ defined by complementary recovery is in fact unique
		\begin{theorem}[Uniqueness of complementary von Neumann algebra]
			Suppose $V$ is an encoding isometry and $A$ is a subregion inducing $\mathcal{H}=\mathcal{H}_{A}\otimes\mathcal{H}_{\ol{A}}$. Let $M=V^{\dagger}(\mathcal{L}(\mathcal{H}_{A})\otimes I_{\ol{A}})V$ be the image of operators on $\mathcal{H}_{A}$ projected onto $\mathcal{H}_{L}$. If $M$ is a von Neumann algebra, then is is the unique von Neumann algebra satisfying complementary recovery on $V$ and $A$. If $M$ is not a von Neumann algebra, then no von Neumann algebra satisfying complementary recovery exists.
		\end{theorem}
		\begin{proof}
			The theorem essentially has two conditions:
			\begin{itemize}
				\item \textbf{Existence}: If $M=V^{\dagger}(\mathcal{L}(\mathcal{H}_{A})\otimes I_{\ol{A}})V$ is a von Neumann algebra, then $(V,A,M)$ have complementary recovery.
				\item \textbf{Uniqueness}: If $N\subseteq V^{\dagger}(\mathcal{L}(\mathcal{H}_{A})\otimes I_{\ol{A}})V$ is a von Neumann algebra, then $(V,A,N)$ do not have complementary recovery.
			\end{itemize}
			which we prove in turn.\\
			\textbf{Existence}: Assume $M=V^{\dagger}(\mathcal{L}(\mathcal{H}_{A})\otimes I_{\ol{A}})V$ is a von Neumann algebra, so the commutant $M'$ is well defined. The first condition of complementary recovery ($M$ is correctable on $A$ with respect to $V$) is trivial by definition of $M$. Also note that by the bicommutant theorem:
			\begin{equation}
				\mdef\subseteq M=M''
			\end{equation}
			so by definition of privacy, $M'$ is private from $A$ with respect to $V$. So by theorem (2.1), $M'$ is correctable from $\ol{A}$ with respect to $V$, which is the second condition for complementary recovery.\\
			\textbf{Uniqueness}: Suppose $N\subsetneq\mdef$ is a von Neumann algebra which is correctable from $A$ but not equal to the full set of correctable operators. For sake of contradiction, assume $(V,A,N)$ have complementary recovery. By the second condition of complementary recovery, $N'$ is correctable from $\ol{A}$ with respect to $V$, so by theorem (2.1), $N'$ is private from $A$ with respect to $V$. This means:
			\begin{equation}
				\mdef\subseteq N''=N
			\end{equation}
			by the bicommutant theorem, and so we have
			\begin{equation}
				N\subsetneq \mdef\subseteq N
			\end{equation}
			which is a contradiction since $N\subseteq N$ is a tautology, so such an $N$ cannot exist.
		\end{proof}
		\item Complementarity seems like a natural property for quantum error correcting codes to have
		\item It turns out that complementary recovery implies an RT formula, which is surprising
		\item The fact that von Neumann algebras with complementary recovery can fail to exist is quite non-trivial - to this end, we give an example of a code without complementary recovery
		\begin{example}
			We let the physical space be that of two qubits: $\mathcal{H}=\mathcal{H}_{2}\otimes\mathcal{H}_{2}$, and let the logical space be a qutrit $\mathcal{H}_{L}=\text{span}(\ket{0},\ket{1},\ket{2})$. We denote the first qubit of $\mathcal{H}$ by $A$, and then give encoding isometry
			\begin{equation}
				V=\ketbra{00}{0}+\ketbra{01}{1}+\ketbra{10}{2}
			\end{equation}
			The set of correctable operators is then
			\begin{equation}
				\mdef=\begin{pmatrix}
					a&b&0\\c&d&0\\0&0&a
				\end{pmatrix}
			\end{equation}
			where $a,b,c,d\in\mathbb{C}$. This set itself is not closed under multiplication, so cannot be a von Neumann algebra, so we take $M$ to be the largest von Neumann algebra contained within $\mdef$, which is given by
			\begin{equation}
				M=\begin{pmatrix}
					a&0&0\\0&d&0\\0&0&a
				\end{pmatrix}
			\end{equation}
			$(V,A,M)$ satisfy the first condition of complementary recovery ($M$ is correctable from $A$ with respect to $V$), but not the second ($M'$ is correctable from $\ol{A}$ with respect to $V$), since
			\begin{equation}
				\begin{pmatrix}
					a&0&b\\0&d&0\\c&0&e
				\end{pmatrix}
				=M'\nsubseteq V^{\dagger}(I_{A}\otimes\mathcal{L}(\mathcal{H}_{\ol{A}}))V=\begin{pmatrix}
					a&0&b\\0&a&0\\c&0&e
				\end{pmatrix}
			\end{equation}
			If we had chosen $M$ to be a smaller subalgebra, then $M'$ would be larger and in particular would contain the above. But since $M'$ is already not contained in the relevant object, there cannot exist a von Neumann algebra with complementary recovery.
		\end{example}
	\end{itemize}
	\subsection{The RT Formula and Properties}
	\begin{itemize}
		\item In the language we have so far, we can \textit{define} what an RT formula is, with no reference to holography
		\begin{definition}[RT Formula]
			Say $V$ is an encoding isometry, $A$ is a subregion, and $M$ is a von Neumann algebra on $\mathcal{H}_{L}$. We say $(V,A,M)$ has an \textbf{RT formula} if there exists an \textbf{area operator} $L\in\mathcal{L}(\mathcal{H}_{L})$ such that for any state $\rho$ with support on $\mathcal{H}_{L}$, we have:
			\begin{equation}
				S(\text{Tr}_{\ol{A}}(V\rho V^{\dagger}))=S(M,\rho)+\text{Tr}(\rho L)
			\end{equation}
			If $L\propto I$, we say $(V,A,M)$ has a trivial RT formula.
		\end{definition}
		\item The existence of an RT formula in a code is closely connected to complementary recovery
		\item Recalling that a von Neumann algebra implies a Wedderburn decomposition on the Hilbert space it acts on ($\mathcal{H}=\oplus_{\alpha}(\mathcal{H}_{A_{\alpha}}\otimes\mathcal{H}_{\ol{A}_{\alpha}})$), Pollack finds that when a von Neumann algebra is correctable from $A$ with respect to $V$, then the Hilbert space $\mathcal{H}_{A}$ also decomposes, where the decomposition of this and of $\mathcal{H}_{L}$ are related
		\item This is formalised by the following lemma - note the similarity to Harlow 3
		\begin{theorem}
			Suppose $V\;:\,\mathcal{H}_{L}\to\mathcal{H}$ is an encoding isometry, $A$ is a subregion inducing $\mathcal{H}=\mathcal{H}_{A}\otimes\mathcal{H}_{\ol{A}}$, and $M$ is a von Neumann algebra on $\mathcal{H}_{L}$ which is correctable from $A$ with respect to $V$.\\
			Suppose $M$ induces the decomposition $\mathcal{H}_{L}=\oplus_{\alpha}(\mathcal{H}_{L_{\alpha}}\otimes\mathcal{H}_{\ol{L}_{\alpha}})$, so that 
			\begin{equation}
				M=\oplus_{\alpha}(\mathcal{L}(\mathcal{H}_{L_{\alpha}})\otimes I_{\ol{L}_{\alpha}})
			\end{equation}
			Set $\{\ket{\alpha,i,j}\}$ to be an orthonormal basis of $\mathcal{H}_{L}$ which is compatible with $M$, so $\alpha$ enumerates the diagonal blocks and within each block we have $\ket{\alpha,i,j}=\ket{i_{\alpha}}_{L_{\alpha}}\otimes\ket{j_{\alpha}}_{\ol{L}_{\alpha}}$ where $\{\ket{i_{\alpha}}_{L_{\alpha}}\}$ and $\{\ket{j_{\alpha}}_{\ol{L}_{\alpha}}\}$ are orthonormal bases for $\mathcal{H}_{L_{\alpha}}$ and $\mathcal{H}_{\ol{L}_{\alpha}}$ respectively.\\
			Then, there exists a factorisation $\mathcal{H}_{A}=\oplus_{\alpha}(\mathcal{H}_{A_{1}^{\alpha}}\otimes\mathcal{H}_{A_{2}^{\alpha}})\oplus\mathcal{H}_{A_{3}}$ and a unitary $U_{A}$ on $\mathcal{H}_{A}$ such that the state $(U_{A}\otimes I_{\ol{A}})V\ket{\alpha,i,j}$ factorises as
			\begin{equation}
				(U_{A}\otimes I_{\ol{A}})V\ket{\alpha,i,j}=\ket{\psi_{\alpha,i}}_{A_{1}^{\alpha}}\otimes\ket{\chi_{\alpha,j}}_{A_{2}^{\alpha}\ol{A}}
			\end{equation}
			where $\ket{\psi_{\alpha,i}}$ is independent of $j$ and $\ket{\chi_{\alpha,j}}$ is independent of $i$.
		\end{theorem}
		\item Note that this is essentially Harlow 3 in the language of Pollack, and this theorem specifically says $(1)\implies(4)$
		\item To see why this is true, recall the first condition of Harlow's: for any operator $\tilde{O}$ on $M$, there exists an operator $O_{A}$ on $\mathcal{H}_{A}$ with the same action on $\Hcode$, so for any $\ket{\tilde{\psi}}\in\Hcode$, we have $O_{A}\ket{\tilde{\psi}}=\tilde{O}\ket{\tilde{\psi}}$
		\item This is just the statement that $M$ is correctable from $A$ with respect to $V$!
		\item Indeed, if $M$ is correctable from $A$ with respect to $V$, then for every $O_{L}\in M$ there is a corresponding $O_{A}\in\mathcal{L}(\mathcal{H}_{A})$ such that $O_{L}=V^{\dagger}(O_{A}\otimes I_{\ol{A}})V$
		\item Therefore, for any $\ket{\psi}\in\mathcal{H}_{L}$ and corresponding $V\ket{\psi}\in V(\mathcal{H}_{L})$, we clearly have:
		\begin{equation}
			(O_{A}\otimes I_{\ol{A}})V\ket{\psi}=(VO_{L}V^{\dagger})(V\ket{\psi})=O_{L}\ket{\psi}
		\end{equation}
		which is just Harlow's second condition as claimed
		\item The proof therefore straightforwardly adapts
		\item The next thing Pollack does is show that complementary recovery implies a `two-sided RT formula'; that is, $(V,A,M)$ and $(V,\ol{A},M')$ have an RT formula with the same area operator
	\end{itemize}
\end{document}
